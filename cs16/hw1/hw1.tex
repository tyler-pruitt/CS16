\begin{longtable}[]{@{}cl@{}}
\toprule
\endhead
\textbf{Due:} & Tuesday, January 14, 2021 (11:59 PM PST)\tabularnewline
\textbf{Points:} & 100\tabularnewline
\textbf{Name:} &
\texttt{\_\_\_\_\_\_\_\_\_\_\_\_\_\_\_\_\_\_\_\_\_\_\_\_\_\_\_\_\_\_\_\_\_\_\_\_\_\_\_\_\_\_\_\_\_\_\_\_\_\_\_\_\_\_\_}\tabularnewline
\textbf{Homework buddy:} &
\texttt{\_\_\_\_\_\_\_\_\_\_\_\_\_\_\_\_\_\_\_\_\_\_\_\_\_\_\_\_\_\_\_\_\_\_\_\_\_\_\_\_\_\_\_\_\_\_\_\_\_\_\_\_\_\_\_}\tabularnewline
\bottomrule
\end{longtable}

\begin{itemize}
\tightlist
\item
  You may collaborate on this homework with \textbf{at most} one person,
  an optional ``homework buddy.''
\item
  \textbf{Submission instructions:} All questions are to be written
  (either by hand or typed) \emph{in the provided spaces} and turned in
  as a single PDF on Gradescope. If you submit handwritten solutions
  write legibly. We reserve the right to give 0 points to answers we
  cannot read.
\end{itemize}

\begin{enumerate}
\def\labelenumi{\arabic{enumi}.}
\tightlist
\item
  (5 points) Not including any comments that may appear, what are the
  first two lines that typically begin a C++ program that is either
  going to output on the screen and/or read input from the keyboard?
\end{enumerate}

\begin{verbatim}

\end{verbatim}

\begin{enumerate}
\def\labelenumi{\arabic{enumi}.}
\setcounter{enumi}{1}
\tightlist
\item
  (5 points) What statement is the recommended way to end a C++ program?
\end{enumerate}

\begin{verbatim}

\end{verbatim}

\begin{enumerate}
\def\labelenumi{\arabic{enumi}.}
\setcounter{enumi}{2}
\item
  (15 points) The textbook author describes the difference between
  \textbf{syntax errors} and \textbf{logic errors}, as well as the
  difference between compiler output that produces \textbf{error
  messages} vs \textbf{warning messages}. Briefly explain each of the
  items below in a way that makes the \emph{differences} among them
  clear.

  \begin{enumerate}
  \def\labelenumii{\alph{enumii}.}
  \tightlist
  \item
    (5 points) Syntax error that results in an \emph{error} message:
  \end{enumerate}

\begin{verbatim}



\end{verbatim}

  \begin{enumerate}
  \def\labelenumii{\alph{enumii}.}
  \setcounter{enumii}{1}
  \tightlist
  \item
    (5 points) Syntax error that results in a \emph{warning} message:
  \end{enumerate}

\begin{verbatim}



\end{verbatim}

  \begin{enumerate}
  \def\labelenumii{\alph{enumii}.}
  \setcounter{enumii}{2}
  \tightlist
  \item
    (5 points) Logic errors:
  \end{enumerate}

\begin{verbatim}



\end{verbatim}
\item
  (5 points) Assuming the variable \texttt{age} has already been
  declared as \texttt{int\ age;} what single statement of code will read
  in a value for \texttt{age} from the user?
\end{enumerate}

\begin{verbatim}

\end{verbatim}

\begin{enumerate}
\def\labelenumi{\arabic{enumi}.}
\setcounter{enumi}{4}
\tightlist
\item
  (10 points) Assuming the variable \texttt{balance} has already been
  declared as \texttt{int\ balance;} write two code statements that will
  ask (prompt) the user for a value for \texttt{balance}, and then read
  in the value of \texttt{balance}.
\end{enumerate}

\begin{verbatim}


\end{verbatim}

\begin{enumerate}
\def\labelenumi{\arabic{enumi}.}
\setcounter{enumi}{5}
\tightlist
\item
  (5 points) The textbook describes \textbf{C++11} on page 27. Briefly,
  what is C++11? (A one sentence answer is good enough.)
\end{enumerate}

\begin{verbatim}



\end{verbatim}

\begin{enumerate}
\def\labelenumi{\arabic{enumi}.}
\setcounter{enumi}{6}
\item
  (10 points) The book talks about the 5 important components of a
  computer: (1) processor, (2) input devices, (3) output devices, (4)
  main memory, (5) secondary memory. It also talks about two important
  pieces of software: compilers and operating systems. What of the above
  is primarily responsible for each of the following tasks? Write
  ``none'' if none of the options apply.

  \begin{enumerate}
  \def\labelenumii{\alph{enumii}.}
  \tightlist
  \item
    (2 points) Executes a program stored in main memory.
  \end{enumerate}

\begin{verbatim}

\end{verbatim}

  \begin{enumerate}
  \def\labelenumii{\alph{enumii}.}
  \setcounter{enumii}{1}
  \tightlist
  \item
    (2 points) Allocates the computer's resources to different tasks.
  \end{enumerate}

\begin{verbatim}

\end{verbatim}

  \begin{enumerate}
  \def\labelenumii{\alph{enumii}.}
  \setcounter{enumii}{2}
  \tightlist
  \item
    (2 points) Stores a program while it is being executed.
  \end{enumerate}

\begin{verbatim}

\end{verbatim}

  \begin{enumerate}
  \def\labelenumii{\alph{enumii}.}
  \setcounter{enumii}{3}
  \tightlist
  \item
    (2 points) Stores a program when it is not being executed.
  \end{enumerate}

\begin{verbatim}

\end{verbatim}

  \begin{enumerate}
  \def\labelenumii{\alph{enumii}.}
  \setcounter{enumii}{4}
  \tightlist
  \item
    (2 points) Converts a program written in a high-level language to
    another high-level language.
  \end{enumerate}

\begin{verbatim}

\end{verbatim}
\item
  (5 points) In one sentence, what is the role of a \emph{compiler}?
\end{enumerate}

\begin{verbatim}



\end{verbatim}

\begin{enumerate}
\def\labelenumi{\arabic{enumi}.}
\setcounter{enumi}{8}
\tightlist
\item
  (5 points) What is \emph{object code} (and how is it different from
  C++ code)?
\end{enumerate}

\begin{verbatim}



\end{verbatim}

\begin{enumerate}
\def\labelenumi{\arabic{enumi}.}
\setcounter{enumi}{9}
\tightlist
\item
  (10 points) If the following statement were in a C++ program, what
  would it do?
\end{enumerate}

\begin{verbatim}
cout >> "A penny saved";
\end{verbatim}

\begin{verbatim}



\end{verbatim}

\begin{enumerate}
\def\labelenumi{\arabic{enumi}.}
\setcounter{enumi}{10}
\tightlist
\item
  (10 points) If the following statement were in a C++ program, what
  would it do?
\end{enumerate}

\begin{verbatim}
cout << "Is a penny earned.";
\end{verbatim}

\begin{verbatim}



\end{verbatim}

\begin{enumerate}
\def\labelenumi{\arabic{enumi}.}
\setcounter{enumi}{11}
\tightlist
\item
  (15 points) Complete this C++ program (as indicated by the comments)
  designed to calculate the area and circumference of a circle. The
  program gets the \emph{diameter} parameter from the user and then
  prints out statements that say:
\end{enumerate}

\begin{verbatim}
The area of this circle is: <RESULT HERE>
The circumference of this circle is: <RESULT HERE>
\end{verbatim}

Notes: (1) In the output replace
\texttt{\textless{}RESULT\ HERE\textgreater{}} with the appropriate
results. (2) Use the C++ \texttt{const} keyword to declare a value for
pi (\(\pi\)). (3) Your code must be syntactically correct (i.e.~it
should compile without error).

\begin{verbatim}
#include <iostream>
using namespace std;

int main() {
    // declare the variables here



    // calculate the results here



    // print statements here



    // end program

}
\end{verbatim}
