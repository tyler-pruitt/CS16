% Options for packages loaded elsewhere
\PassOptionsToPackage{unicode}{hyperref}
\PassOptionsToPackage{hyphens}{url}
%
\documentclass[
]{article}
\usepackage{lmodern}
\usepackage{amssymb,amsmath}
\usepackage{ifxetex,ifluatex}
\ifnum 0\ifxetex 1\fi\ifluatex 1\fi=0 % if pdftex
  \usepackage[T1]{fontenc}
  \usepackage[utf8]{inputenc}
  \usepackage{textcomp} % provide euro and other symbols
\else % if luatex or xetex
  \usepackage{unicode-math}
  \defaultfontfeatures{Scale=MatchLowercase}
  \defaultfontfeatures[\rmfamily]{Ligatures=TeX,Scale=1}
\fi
% Use upquote if available, for straight quotes in verbatim environments
\IfFileExists{upquote.sty}{\usepackage{upquote}}{}
\IfFileExists{microtype.sty}{% use microtype if available
  \usepackage[]{microtype}
  \UseMicrotypeSet[protrusion]{basicmath} % disable protrusion for tt fonts
}{}
\makeatletter
\@ifundefined{KOMAClassName}{% if non-KOMA class
  \IfFileExists{parskip.sty}{%
    \usepackage{parskip}
  }{% else
    \setlength{\parindent}{0pt}
    \setlength{\parskip}{6pt plus 2pt minus 1pt}}
}{% if KOMA class
  \KOMAoptions{parskip=half}}
\makeatother
\usepackage{xcolor}
\IfFileExists{xurl.sty}{\usepackage{xurl}}{} % add URL line breaks if available
\IfFileExists{bookmark.sty}{\usepackage{bookmark}}{\usepackage{hyperref}}
\hypersetup{
  pdftitle={Homework 8: Classes++},
  pdfauthor={CS16 - Winter 2021},
  hidelinks,
  pdfcreator={LaTeX via pandoc}}
\urlstyle{same} % disable monospaced font for URLs
\usepackage{color}
\usepackage{fancyvrb}
\newcommand{\VerbBar}{|}
\newcommand{\VERB}{\Verb[commandchars=\\\{\}]}
\DefineVerbatimEnvironment{Highlighting}{Verbatim}{commandchars=\\\{\}}
% Add ',fontsize=\small' for more characters per line
\newenvironment{Shaded}{}{}
\newcommand{\AlertTok}[1]{\textcolor[rgb]{1.00,0.00,0.00}{\textbf{#1}}}
\newcommand{\AnnotationTok}[1]{\textcolor[rgb]{0.38,0.63,0.69}{\textbf{\textit{#1}}}}
\newcommand{\AttributeTok}[1]{\textcolor[rgb]{0.49,0.56,0.16}{#1}}
\newcommand{\BaseNTok}[1]{\textcolor[rgb]{0.25,0.63,0.44}{#1}}
\newcommand{\BuiltInTok}[1]{#1}
\newcommand{\CharTok}[1]{\textcolor[rgb]{0.25,0.44,0.63}{#1}}
\newcommand{\CommentTok}[1]{\textcolor[rgb]{0.38,0.63,0.69}{\textit{#1}}}
\newcommand{\CommentVarTok}[1]{\textcolor[rgb]{0.38,0.63,0.69}{\textbf{\textit{#1}}}}
\newcommand{\ConstantTok}[1]{\textcolor[rgb]{0.53,0.00,0.00}{#1}}
\newcommand{\ControlFlowTok}[1]{\textcolor[rgb]{0.00,0.44,0.13}{\textbf{#1}}}
\newcommand{\DataTypeTok}[1]{\textcolor[rgb]{0.56,0.13,0.00}{#1}}
\newcommand{\DecValTok}[1]{\textcolor[rgb]{0.25,0.63,0.44}{#1}}
\newcommand{\DocumentationTok}[1]{\textcolor[rgb]{0.73,0.13,0.13}{\textit{#1}}}
\newcommand{\ErrorTok}[1]{\textcolor[rgb]{1.00,0.00,0.00}{\textbf{#1}}}
\newcommand{\ExtensionTok}[1]{#1}
\newcommand{\FloatTok}[1]{\textcolor[rgb]{0.25,0.63,0.44}{#1}}
\newcommand{\FunctionTok}[1]{\textcolor[rgb]{0.02,0.16,0.49}{#1}}
\newcommand{\ImportTok}[1]{#1}
\newcommand{\InformationTok}[1]{\textcolor[rgb]{0.38,0.63,0.69}{\textbf{\textit{#1}}}}
\newcommand{\KeywordTok}[1]{\textcolor[rgb]{0.00,0.44,0.13}{\textbf{#1}}}
\newcommand{\NormalTok}[1]{#1}
\newcommand{\OperatorTok}[1]{\textcolor[rgb]{0.40,0.40,0.40}{#1}}
\newcommand{\OtherTok}[1]{\textcolor[rgb]{0.00,0.44,0.13}{#1}}
\newcommand{\PreprocessorTok}[1]{\textcolor[rgb]{0.74,0.48,0.00}{#1}}
\newcommand{\RegionMarkerTok}[1]{#1}
\newcommand{\SpecialCharTok}[1]{\textcolor[rgb]{0.25,0.44,0.63}{#1}}
\newcommand{\SpecialStringTok}[1]{\textcolor[rgb]{0.73,0.40,0.53}{#1}}
\newcommand{\StringTok}[1]{\textcolor[rgb]{0.25,0.44,0.63}{#1}}
\newcommand{\VariableTok}[1]{\textcolor[rgb]{0.10,0.09,0.49}{#1}}
\newcommand{\VerbatimStringTok}[1]{\textcolor[rgb]{0.25,0.44,0.63}{#1}}
\newcommand{\WarningTok}[1]{\textcolor[rgb]{0.38,0.63,0.69}{\textbf{\textit{#1}}}}
\usepackage{longtable,booktabs}
% Correct order of tables after \paragraph or \subparagraph
\usepackage{etoolbox}
\makeatletter
\patchcmd\longtable{\par}{\if@noskipsec\mbox{}\fi\par}{}{}
\makeatother
% Allow footnotes in longtable head/foot
\IfFileExists{footnotehyper.sty}{\usepackage{footnotehyper}}{\usepackage{footnote}}
\makesavenoteenv{longtable}
\setlength{\emergencystretch}{3em} % prevent overfull lines
\providecommand{\tightlist}{%
  \setlength{\itemsep}{0pt}\setlength{\parskip}{0pt}}
\setcounter{secnumdepth}{-\maxdimen} % remove section numbering
\ifluatex
  \usepackage{selnolig}  % disable illegal ligatures
\fi

\title{Homework 8: Classes++}
\author{CS16 - Winter 2021}
\date{}

\begin{document}
\maketitle

\begin{longtable}[]{@{}cl@{}}
\toprule
\endhead
\textbf{Due:} & Thursday, March 4, 2021 (11:59 PM PST)\tabularnewline
\textbf{Points:} & 40\tabularnewline
\textbf{Name:} &
\texttt{\_\_\_\_\_\_\_\_\_\_\_\_\_\_\_\_\_\_\_\_\_\_\_\_\_\_\_\_\_\_\_\_\_\_\_\_\_\_\_\_\_\_\_\_\_\_\_\_\_\_\_\_\_\_\_}\tabularnewline
\textbf{Homework buddy:} &
\texttt{\_\_\_\_\_\_\_\_\_\_\_\_\_\_\_\_\_\_\_\_\_\_\_\_\_\_\_\_\_\_\_\_\_\_\_\_\_\_\_\_\_\_\_\_\_\_\_\_\_\_\_\_\_\_\_}\tabularnewline
\bottomrule
\end{longtable}

\begin{itemize}
\tightlist
\item
  You may collaborate on this homework with \textbf{at most} one person,
  an optional ``homework buddy.''
\item
  \textbf{Submission instructions:} All questions are to be written
  (either by hand or typed) \emph{in the provided spaces} and turned in
  as a single PDF on Gradescope. If you submit handwritten solutions
  write legibly. We reserve the right to give 0 points to answers we
  cannot read. When you submit your answer on Gradescope, \textbf{be
  sure to select which portions of your answer correspond to which
  problem} and clearly mark on the page itself which problem you are
  answering. We reserve the right to give 0 points to submissions that
  fail to do this.
\end{itemize}

\pagenumbering{gobble}

\begin{enumerate}
\def\labelenumi{\arabic{enumi}.}
\tightlist
\item
  (4 points) According to lecture and the textbook, what are the rules
  of class definition in order to make a class an abstract data type
  (ADT)?
\end{enumerate}

\begin{verbatim}





\end{verbatim}

\begin{enumerate}
\def\labelenumi{\arabic{enumi}.}
\setcounter{enumi}{1}
\tightlist
\item
  (2 points) What are derived classes and what mechanism do they use in
  order to fulfill what they need to do?
\end{enumerate}

\begin{verbatim}




\end{verbatim}

\begin{enumerate}
\def\labelenumi{\arabic{enumi}.}
\setcounter{enumi}{2}
\tightlist
\item
  (2 points) Can a derived class directly access by name a private
  member variable of the parent class?
\end{enumerate}

\begin{verbatim}




\end{verbatim}

\begin{enumerate}
\def\labelenumi{\arabic{enumi}.}
\setcounter{enumi}{3}
\tightlist
\item
  (2 points) Suppose the class \texttt{SportsCar} is a publicly derived
  class of a class \texttt{Automobile}. Suppose also that the class
  \texttt{Automobile} has public member functions named
  \texttt{accelerate} and \texttt{addGas}. Will an object of the class
  \texttt{SportsCar} have member functions named \texttt{accelerate} and
  \texttt{addGas}?
\end{enumerate}

\begin{verbatim}


\end{verbatim}

\begin{enumerate}
\def\labelenumi{\arabic{enumi}.}
\setcounter{enumi}{4}
\tightlist
\item
  (14 points) Suppose your program contains the following class
  definition:
\end{enumerate}

\begin{Shaded}
\begin{Highlighting}[]
\KeywordTok{class}\NormalTok{ Automobile \{}
   \KeywordTok{public}\NormalTok{:}
      \DataTypeTok{void}\NormalTok{ set\_price(}\DataTypeTok{double}\NormalTok{ new\_price);}
      \DataTypeTok{void}\NormalTok{ set\_profit(}\DataTypeTok{double}\NormalTok{ new\_profit);}
      \DataTypeTok{double}\NormalTok{ get\_price();}
   \KeywordTok{private}\NormalTok{:}
      \DataTypeTok{double}\NormalTok{ price;}
      \DataTypeTok{double}\NormalTok{ profit;}
      \DataTypeTok{double}\NormalTok{ get\_profit();}
\NormalTok{\};}
\end{Highlighting}
\end{Shaded}

Suppose the main part of your program contains the following declaration
and that the program somehow sets the values of all the member variables
to some values:

\begin{Shaded}
\begin{Highlighting}[]
\NormalTok{Automobile hyundai, jaguar; }
\end{Highlighting}
\end{Shaded}

Which of the following statements are then \textbf{not} allowed in the
main part of your program and explain \textbf{why}.

\begin{enumerate}
\def\labelenumi{(\alph{enumi})}
\tightlist
\item
  \texttt{hyundai.price\ =\ 4999.99;}
\item
  \texttt{jaguar.set\_price(30000.97);}
\item
  \texttt{double\ a\_price,\ a\_profit;}
\item
  \texttt{a\_price\ =\ jaguar.get\_price();}
\item
  \texttt{a\_profit\ =\ jaguar.get\_profit();}
\item
  \texttt{a\_profit\ =\ hyundai.get\_profit();}
\item
  \texttt{if\ (hyundai\ ==\ jaguar)\ \{hyundai\ =\ jaguar;\}}
\end{enumerate}

\begin{verbatim}
\end{verbatim}

\pagebreak

\begin{enumerate}
\def\labelenumi{\arabic{enumi}.}
\setcounter{enumi}{5}
\tightlist
\item
  (16 points) Suppose your program contains the following class
  definition:
\end{enumerate}

\begin{Shaded}
\begin{Highlighting}[]
\KeywordTok{class}\NormalTok{ TwoNumbers \{}
   \KeywordTok{public}\NormalTok{:}
\NormalTok{      TwoNumbers(}\DataTypeTok{int}\NormalTok{ n1, }\DataTypeTok{int}\NormalTok{ n2);}
\NormalTok{      TwoNumbers(); }\CommentTok{// initializes num1, num2 to 0}
      \DataTypeTok{double}\NormalTok{ sum(); }\CommentTok{// returns sum of num1 \& num2}
      \DataTypeTok{double}\NormalTok{ difference(); }\CommentTok{// returns diff. of num1 from num2}
      \DataTypeTok{double}\NormalTok{ max(); }\CommentTok{// returns larger of num1, num2}
   \KeywordTok{private}\NormalTok{:}
      \DataTypeTok{double}\NormalTok{ num1, num2;}
\NormalTok{\};}
\end{Highlighting}
\end{Shaded}

\begin{enumerate}
\def\labelenumi{\alph{enumi}.}
\tightlist
\item
  (10 points) Given the comments shown, give definitions to all 5 of
  these member functions/constructors:
\end{enumerate}

\begin{verbatim}
\end{verbatim}

\pagebreak

\begin{enumerate}
\def\labelenumi{\alph{enumi}.}
\setcounter{enumi}{1}
\tightlist
\item
  (2 points) Consider these instructions in \texttt{main()}:
\end{enumerate}

\begin{Shaded}
\begin{Highlighting}[]
\NormalTok{TwoNumbers thisOne, thatOne(}\DecValTok{5}\NormalTok{,}\DecValTok{7}\NormalTok{);}
\NormalTok{thisOne.num1++;}
\NormalTok{thisOne.num2 {-}= }\DecValTok{7}\NormalTok{;}
\NormalTok{thatOne.num2 = thatOne.sum() + thisOne.difference();}
\NormalTok{cout \textless{}\textless{} thisOne.max() / thatOne.max();}
\end{Highlighting}
\end{Shaded}

Explain all the reasons \textbf{why} this code will not compile.

\begin{verbatim}



\end{verbatim}

\begin{enumerate}
\def\labelenumi{\alph{enumi}.}
\setcounter{enumi}{2}
\tightlist
\item
  (2 points) What would you change to the class definition to make this
  code compile?
\end{enumerate}

\begin{verbatim}



\end{verbatim}

\begin{enumerate}
\def\labelenumi{\alph{enumi}.}
\setcounter{enumi}{3}
\tightlist
\item
  (2 points) When you fix it, what would these instructions do?
\end{enumerate}

\begin{verbatim}
\end{verbatim}

\end{document}
