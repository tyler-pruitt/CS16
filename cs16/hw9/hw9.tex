% Options for packages loaded elsewhere
\PassOptionsToPackage{unicode}{hyperref}
\PassOptionsToPackage{hyphens}{url}
%
\documentclass[
]{article}
\usepackage{amsmath,amssymb}
\usepackage{lmodern}
\usepackage{ifxetex,ifluatex}
\ifnum 0\ifxetex 1\fi\ifluatex 1\fi=0 % if pdftex
  \usepackage[T1]{fontenc}
  \usepackage[utf8]{inputenc}
  \usepackage{textcomp} % provide euro and other symbols
\else % if luatex or xetex
  \usepackage{unicode-math}
  \defaultfontfeatures{Scale=MatchLowercase}
  \defaultfontfeatures[\rmfamily]{Ligatures=TeX,Scale=1}
\fi
% Use upquote if available, for straight quotes in verbatim environments
\IfFileExists{upquote.sty}{\usepackage{upquote}}{}
\IfFileExists{microtype.sty}{% use microtype if available
  \usepackage[]{microtype}
  \UseMicrotypeSet[protrusion]{basicmath} % disable protrusion for tt fonts
}{}
\makeatletter
\@ifundefined{KOMAClassName}{% if non-KOMA class
  \IfFileExists{parskip.sty}{%
    \usepackage{parskip}
  }{% else
    \setlength{\parindent}{0pt}
    \setlength{\parskip}{6pt plus 2pt minus 1pt}}
}{% if KOMA class
  \KOMAoptions{parskip=half}}
\makeatother
\usepackage{xcolor}
\IfFileExists{xurl.sty}{\usepackage{xurl}}{} % add URL line breaks if available
\IfFileExists{bookmark.sty}{\usepackage{bookmark}}{\usepackage{hyperref}}
\hypersetup{
  pdftitle={Homework 9: Recursive functions},
  pdfauthor={CS16 - Winter 2021},
  hidelinks,
  pdfcreator={LaTeX via pandoc}}
\urlstyle{same} % disable monospaced font for URLs
\usepackage{color}
\usepackage{fancyvrb}
\newcommand{\VerbBar}{|}
\newcommand{\VERB}{\Verb[commandchars=\\\{\}]}
\DefineVerbatimEnvironment{Highlighting}{Verbatim}{commandchars=\\\{\}}
% Add ',fontsize=\small' for more characters per line
\newenvironment{Shaded}{}{}
\newcommand{\AlertTok}[1]{\textcolor[rgb]{1.00,0.00,0.00}{\textbf{#1}}}
\newcommand{\AnnotationTok}[1]{\textcolor[rgb]{0.38,0.63,0.69}{\textbf{\textit{#1}}}}
\newcommand{\AttributeTok}[1]{\textcolor[rgb]{0.49,0.56,0.16}{#1}}
\newcommand{\BaseNTok}[1]{\textcolor[rgb]{0.25,0.63,0.44}{#1}}
\newcommand{\BuiltInTok}[1]{#1}
\newcommand{\CharTok}[1]{\textcolor[rgb]{0.25,0.44,0.63}{#1}}
\newcommand{\CommentTok}[1]{\textcolor[rgb]{0.38,0.63,0.69}{\textit{#1}}}
\newcommand{\CommentVarTok}[1]{\textcolor[rgb]{0.38,0.63,0.69}{\textbf{\textit{#1}}}}
\newcommand{\ConstantTok}[1]{\textcolor[rgb]{0.53,0.00,0.00}{#1}}
\newcommand{\ControlFlowTok}[1]{\textcolor[rgb]{0.00,0.44,0.13}{\textbf{#1}}}
\newcommand{\DataTypeTok}[1]{\textcolor[rgb]{0.56,0.13,0.00}{#1}}
\newcommand{\DecValTok}[1]{\textcolor[rgb]{0.25,0.63,0.44}{#1}}
\newcommand{\DocumentationTok}[1]{\textcolor[rgb]{0.73,0.13,0.13}{\textit{#1}}}
\newcommand{\ErrorTok}[1]{\textcolor[rgb]{1.00,0.00,0.00}{\textbf{#1}}}
\newcommand{\ExtensionTok}[1]{#1}
\newcommand{\FloatTok}[1]{\textcolor[rgb]{0.25,0.63,0.44}{#1}}
\newcommand{\FunctionTok}[1]{\textcolor[rgb]{0.02,0.16,0.49}{#1}}
\newcommand{\ImportTok}[1]{#1}
\newcommand{\InformationTok}[1]{\textcolor[rgb]{0.38,0.63,0.69}{\textbf{\textit{#1}}}}
\newcommand{\KeywordTok}[1]{\textcolor[rgb]{0.00,0.44,0.13}{\textbf{#1}}}
\newcommand{\NormalTok}[1]{#1}
\newcommand{\OperatorTok}[1]{\textcolor[rgb]{0.40,0.40,0.40}{#1}}
\newcommand{\OtherTok}[1]{\textcolor[rgb]{0.00,0.44,0.13}{#1}}
\newcommand{\PreprocessorTok}[1]{\textcolor[rgb]{0.74,0.48,0.00}{#1}}
\newcommand{\RegionMarkerTok}[1]{#1}
\newcommand{\SpecialCharTok}[1]{\textcolor[rgb]{0.25,0.44,0.63}{#1}}
\newcommand{\SpecialStringTok}[1]{\textcolor[rgb]{0.73,0.40,0.53}{#1}}
\newcommand{\StringTok}[1]{\textcolor[rgb]{0.25,0.44,0.63}{#1}}
\newcommand{\VariableTok}[1]{\textcolor[rgb]{0.10,0.09,0.49}{#1}}
\newcommand{\VerbatimStringTok}[1]{\textcolor[rgb]{0.25,0.44,0.63}{#1}}
\newcommand{\WarningTok}[1]{\textcolor[rgb]{0.38,0.63,0.69}{\textbf{\textit{#1}}}}
\usepackage{longtable,booktabs,array}
\usepackage{calc} % for calculating minipage widths
% Correct order of tables after \paragraph or \subparagraph
\usepackage{etoolbox}
\makeatletter
\patchcmd\longtable{\par}{\if@noskipsec\mbox{}\fi\par}{}{}
\makeatother
% Allow footnotes in longtable head/foot
\IfFileExists{footnotehyper.sty}{\usepackage{footnotehyper}}{\usepackage{footnote}}
\makesavenoteenv{longtable}
\setlength{\emergencystretch}{3em} % prevent overfull lines
\providecommand{\tightlist}{%
  \setlength{\itemsep}{0pt}\setlength{\parskip}{0pt}}
\setcounter{secnumdepth}{-\maxdimen} % remove section numbering
\ifluatex
  \usepackage{selnolig}  % disable illegal ligatures
\fi

\title{Homework 9: Recursive functions}
\author{CS16 - Winter 2021}
\date{}

\begin{document}
\maketitle

\begin{longtable}[]{@{}cl@{}}
\toprule
\endhead
\textbf{Due:} & Thursday, March 11, 2021 (11:59 PM PST) \\ \addlinespace
\textbf{Points:} & 105 \\ \addlinespace
\textbf{Name:} &
\texttt{\_\_\_\_\_\_\_\_\_\_\_\_\_\_\_\_\_\_\_\_\_\_\_\_\_\_\_\_\_\_\_\_\_\_\_\_\_\_\_\_\_\_\_\_\_\_\_\_\_\_\_\_\_\_\_} \\ \addlinespace
\textbf{Homework buddy:} &
\texttt{\_\_\_\_\_\_\_\_\_\_\_\_\_\_\_\_\_\_\_\_\_\_\_\_\_\_\_\_\_\_\_\_\_\_\_\_\_\_\_\_\_\_\_\_\_\_\_\_\_\_\_\_\_\_\_} \\ \addlinespace
\bottomrule
\end{longtable}

\begin{itemize}
\tightlist
\item
  You may collaborate on this homework with \textbf{at most} one person,
  an optional ``homework buddy.''
\item
  \textbf{Submission instructions:} All questions are to be written
  (either by hand or typed) \emph{in the provided spaces} and turned in
  as a single PDF on Gradescope. If you submit handwritten solutions
  write legibly. We reserve the right to give 0 points to answers we
  cannot read. When you submit your answer on Gradescope, \textbf{be
  sure to select which portions of your answer correspond to which
  problem} and clearly mark on the page itself which problem you are
  answering. We reserve the right to give 0 points to submissions that
  fail to do this.
\end{itemize}

\pagenumbering{gobble}

\begin{enumerate}
\def\labelenumi{\arabic{enumi}.}
\tightlist
\item
  (3 points) How does a recursive function know when to stop recursing?
\end{enumerate}

\begin{verbatim}



\end{verbatim}

\begin{enumerate}
\def\labelenumi{\arabic{enumi}.}
\setcounter{enumi}{1}
\tightlist
\item
  (4 points) What is a stack overflow? When can it occur? What are the
  consequences?
\end{enumerate}

\begin{verbatim}



\end{verbatim}

\begin{enumerate}
\def\labelenumi{\arabic{enumi}.}
\setcounter{enumi}{2}
\tightlist
\item
  (3 points) What is a LIFO scheme and how does it relate to stacks?
\end{enumerate}

\begin{verbatim}

\end{verbatim}

\begin{enumerate}
\def\labelenumi{\arabic{enumi}.}
\setcounter{enumi}{3}
\tightlist
\item
  (15 points) Write a definition of a recursive function that finds the
  sum of the odd integers in the first \(n\) numbers. Example, if
  \(n = 8\), then the sum is: \((1 + 3 + 5 + 7) = 16\).
\end{enumerate}

\begin{Shaded}
\begin{Highlighting}[]
\DataTypeTok{int}\NormalTok{ sumOdds(}\DataTypeTok{int}\NormalTok{ n)}
\NormalTok{\{}


\end{Highlighting}
\end{Shaded}

\pagebreak

\begin{enumerate}
\def\labelenumi{\arabic{enumi}.}
\setcounter{enumi}{4}
\tightlist
\item
  (25 points) Write a definition of a \emph{recursive} function that
  finds the \(n\)th element in the following arithmetic numerical
  sequence: 3, 11, 27, 59, 123, \ldots{} I've started the program for
  you below, but it is missing the definition. \emph{Hint}: First,
  figure out the recursive pattern as a linear equation,
  i.e.~\(a_n = x * a_{n-1} + y\). You also have to identify the base
  case. Example outputs would look like this (there is no repeating
  loop---these are 2 separate runs):
\end{enumerate}

\begin{verbatim}
Which element of the sequence would you like to know? 4
Element number 4 in the sequence is 59.

Which element of the sequence would you like to know? 7
Element number 7 in the sequence is 507.
\end{verbatim}

\begin{Shaded}
\begin{Highlighting}[]
\PreprocessorTok{\#include }\ImportTok{\textless{}iostream\textgreater{}}
\KeywordTok{using} \KeywordTok{namespace}\NormalTok{ std;}
\DataTypeTok{int}\NormalTok{ RecursiveFunc(}\DataTypeTok{int}\NormalTok{ num);}
\DataTypeTok{int}\NormalTok{ main() \{}
   \DataTypeTok{int}\NormalTok{ elementN;}
\NormalTok{   cout \textless{}\textless{} “Which element of the sequence would you like to know? ”;}
\NormalTok{   cin \textgreater{}\textgreater{} elementN;}
\NormalTok{   cout \textless{}\textless{} “Element number }\DecValTok{4}\NormalTok{ in the sequence is ” }
\NormalTok{        \textless{}\textless{} RecursiveFunc(elementN) \textless{}\textless{} endl;}
   \ControlFlowTok{return} \DecValTok{0}\NormalTok{;}
\NormalTok{\}}

\CommentTok{//DEFINITION HERE:}
\end{Highlighting}
\end{Shaded}

\pagebreak

\begin{enumerate}
\def\labelenumi{\arabic{enumi}.}
\setcounter{enumi}{5}
\item
  (25 points) The Fibonacci sequence is defined as a numerical sequence
  of integers that are the sum of the previous 2 integers. Starting with
  0 and 1, the sequence becomes: \(0, 1, 1, 2, 3, 5, 8, 13, ...\)

  Write the definition of 2 functions: one recursive called
  \texttt{Fibo} (it finds the Nth element in a Fibonacci series) and one
  non-recursive called \texttt{SFS} that calls \texttt{Fibo}.

  \texttt{SFS} has an integer argument \texttt{n}. The pre-condition is
  that \texttt{n} is assumed to be smaller than 256. The post-condition
  is that SFS prints out all the squares of the Fibonacci sequence of
  the first \texttt{n} elements. For example, calling this line in
  \texttt{main} (just like this): \texttt{SFS(7);} will print to
  standard out: \texttt{0\ 1\ 1\ 4\ 9\ 25\ 64}

  (You may assume \texttt{\textless{}cmath\textgreater{}} is already
  included.)
\end{enumerate}

\pagebreak

\begin{enumerate}
\def\labelenumi{\arabic{enumi}.}
\setcounter{enumi}{6}
\tightlist
\item
  (25 points) Write a definition of a recursive function that counts the
  number of the letter `a' or `A' (i.e.~either upper-case or lower-case)
  in a string. Specifically, given a string variable, \texttt{sentence},
  when we pass that into a function \texttt{CountA(sentence)}, the
  function returns an integer that's a count of the number of the letter
  `a' or `A' in the variable \texttt{sentence}. This function
  \textbf{must} be a recursive one and cannot contain a loop.
\end{enumerate}

\begin{verbatim}
\end{verbatim}

\pagebreak

\begin{enumerate}
\def\labelenumi{\arabic{enumi}.}
\setcounter{enumi}{7}
\item
  (5 points) The CS department is currently restructuring its
  lower-division curriculum (CS 16, 24, and 32). We want to gather
  feedback on the experience students are having in the current CS 16
  curriculum. Please take the survey found in the following link. Your
  response will be recorded anonymously. To receive credit for this,
  please attach some proof that you took this survey (e.g., a screenshot
  at the end of survey).

  \texttt{https://ucsb.co1.qualtrics.com/jfe/form/SV\_0qVnEcqYnx9AJkV}
\end{enumerate}

\end{document}
