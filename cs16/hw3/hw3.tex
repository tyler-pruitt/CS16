% Options for packages loaded elsewhere
\PassOptionsToPackage{unicode}{hyperref}
\PassOptionsToPackage{hyphens}{url}
%
\documentclass[
]{article}
\usepackage{lmodern}
\usepackage{amssymb,amsmath}
\usepackage{ifxetex,ifluatex}
\ifnum 0\ifxetex 1\fi\ifluatex 1\fi=0 % if pdftex
  \usepackage[T1]{fontenc}
  \usepackage[utf8]{inputenc}
  \usepackage{textcomp} % provide euro and other symbols
\else % if luatex or xetex
  \usepackage{unicode-math}
  \defaultfontfeatures{Scale=MatchLowercase}
  \defaultfontfeatures[\rmfamily]{Ligatures=TeX,Scale=1}
\fi
% Use upquote if available, for straight quotes in verbatim environments
\IfFileExists{upquote.sty}{\usepackage{upquote}}{}
\IfFileExists{microtype.sty}{% use microtype if available
  \usepackage[]{microtype}
  \UseMicrotypeSet[protrusion]{basicmath} % disable protrusion for tt fonts
}{}
\makeatletter
\@ifundefined{KOMAClassName}{% if non-KOMA class
  \IfFileExists{parskip.sty}{%
    \usepackage{parskip}
  }{% else
    \setlength{\parindent}{0pt}
    \setlength{\parskip}{6pt plus 2pt minus 1pt}}
}{% if KOMA class
  \KOMAoptions{parskip=half}}
\makeatother
\usepackage{xcolor}
\IfFileExists{xurl.sty}{\usepackage{xurl}}{} % add URL line breaks if available
\IfFileExists{bookmark.sty}{\usepackage{bookmark}}{\usepackage{hyperref}}
\hypersetup{
  pdftitle={Homework 3: Functions in C++},
  pdfauthor={CS16 - Winter 2021},
  hidelinks,
  pdfcreator={LaTeX via pandoc}}
\urlstyle{same} % disable monospaced font for URLs
\usepackage{color}
\usepackage{fancyvrb}
\newcommand{\VerbBar}{|}
\newcommand{\VERB}{\Verb[commandchars=\\\{\}]}
\DefineVerbatimEnvironment{Highlighting}{Verbatim}{commandchars=\\\{\}}
% Add ',fontsize=\small' for more characters per line
\newenvironment{Shaded}{}{}
\newcommand{\AlertTok}[1]{\textcolor[rgb]{1.00,0.00,0.00}{\textbf{#1}}}
\newcommand{\AnnotationTok}[1]{\textcolor[rgb]{0.38,0.63,0.69}{\textbf{\textit{#1}}}}
\newcommand{\AttributeTok}[1]{\textcolor[rgb]{0.49,0.56,0.16}{#1}}
\newcommand{\BaseNTok}[1]{\textcolor[rgb]{0.25,0.63,0.44}{#1}}
\newcommand{\BuiltInTok}[1]{#1}
\newcommand{\CharTok}[1]{\textcolor[rgb]{0.25,0.44,0.63}{#1}}
\newcommand{\CommentTok}[1]{\textcolor[rgb]{0.38,0.63,0.69}{\textit{#1}}}
\newcommand{\CommentVarTok}[1]{\textcolor[rgb]{0.38,0.63,0.69}{\textbf{\textit{#1}}}}
\newcommand{\ConstantTok}[1]{\textcolor[rgb]{0.53,0.00,0.00}{#1}}
\newcommand{\ControlFlowTok}[1]{\textcolor[rgb]{0.00,0.44,0.13}{\textbf{#1}}}
\newcommand{\DataTypeTok}[1]{\textcolor[rgb]{0.56,0.13,0.00}{#1}}
\newcommand{\DecValTok}[1]{\textcolor[rgb]{0.25,0.63,0.44}{#1}}
\newcommand{\DocumentationTok}[1]{\textcolor[rgb]{0.73,0.13,0.13}{\textit{#1}}}
\newcommand{\ErrorTok}[1]{\textcolor[rgb]{1.00,0.00,0.00}{\textbf{#1}}}
\newcommand{\ExtensionTok}[1]{#1}
\newcommand{\FloatTok}[1]{\textcolor[rgb]{0.25,0.63,0.44}{#1}}
\newcommand{\FunctionTok}[1]{\textcolor[rgb]{0.02,0.16,0.49}{#1}}
\newcommand{\ImportTok}[1]{#1}
\newcommand{\InformationTok}[1]{\textcolor[rgb]{0.38,0.63,0.69}{\textbf{\textit{#1}}}}
\newcommand{\KeywordTok}[1]{\textcolor[rgb]{0.00,0.44,0.13}{\textbf{#1}}}
\newcommand{\NormalTok}[1]{#1}
\newcommand{\OperatorTok}[1]{\textcolor[rgb]{0.40,0.40,0.40}{#1}}
\newcommand{\OtherTok}[1]{\textcolor[rgb]{0.00,0.44,0.13}{#1}}
\newcommand{\PreprocessorTok}[1]{\textcolor[rgb]{0.74,0.48,0.00}{#1}}
\newcommand{\RegionMarkerTok}[1]{#1}
\newcommand{\SpecialCharTok}[1]{\textcolor[rgb]{0.25,0.44,0.63}{#1}}
\newcommand{\SpecialStringTok}[1]{\textcolor[rgb]{0.73,0.40,0.53}{#1}}
\newcommand{\StringTok}[1]{\textcolor[rgb]{0.25,0.44,0.63}{#1}}
\newcommand{\VariableTok}[1]{\textcolor[rgb]{0.10,0.09,0.49}{#1}}
\newcommand{\VerbatimStringTok}[1]{\textcolor[rgb]{0.25,0.44,0.63}{#1}}
\newcommand{\WarningTok}[1]{\textcolor[rgb]{0.38,0.63,0.69}{\textbf{\textit{#1}}}}
\usepackage{longtable,booktabs}
% Correct order of tables after \paragraph or \subparagraph
\usepackage{etoolbox}
\makeatletter
\patchcmd\longtable{\par}{\if@noskipsec\mbox{}\fi\par}{}{}
\makeatother
% Allow footnotes in longtable head/foot
\IfFileExists{footnotehyper.sty}{\usepackage{footnotehyper}}{\usepackage{footnote}}
\makesavenoteenv{longtable}
\setlength{\emergencystretch}{3em} % prevent overfull lines
\providecommand{\tightlist}{%
  \setlength{\itemsep}{0pt}\setlength{\parskip}{0pt}}
\setcounter{secnumdepth}{-\maxdimen} % remove section numbering
\ifluatex
  \usepackage{selnolig}  % disable illegal ligatures
\fi

\title{Homework 3: Functions in C++}
\author{CS16 - Winter 2021}
\date{}

\begin{document}
\maketitle

\begin{longtable}[]{@{}cl@{}}
\toprule
\endhead
\textbf{Due:} & Thursday, January 28, 2021 (11:59 PM PST)\tabularnewline
\textbf{Points:} & 75\tabularnewline
\textbf{Name:} &
\texttt{\_\_\_\_\_\_\_\_\_\_\_\_\_\_\_\_\_\_\_\_\_\_\_\_\_\_\_\_\_\_\_\_\_\_\_\_\_\_\_\_\_\_\_\_\_\_\_\_\_\_\_\_\_\_\_}\tabularnewline
\textbf{Homework buddy:} &
\texttt{\_\_\_\_\_\_\_\_\_\_\_\_\_\_\_\_\_\_\_\_\_\_\_\_\_\_\_\_\_\_\_\_\_\_\_\_\_\_\_\_\_\_\_\_\_\_\_\_\_\_\_\_\_\_\_}\tabularnewline
\bottomrule
\end{longtable}

\begin{itemize}
\tightlist
\item
  You may collaborate on this homework with \textbf{at most} one person,
  an optional ``homework buddy.''
\item
  \textbf{Submission instructions:} All questions are to be written
  (either by hand or typed) \emph{in the provided spaces} and turned in
  as a single PDF on Gradescope. If you submit handwritten solutions
  write legibly. We reserve the right to give 0 points to answers we
  cannot read. When you submit your answer on Gradescope, \textbf{be
  sure to select which portions of your answer correspond to which
  problem} and clearly mark on the page itself which problem you are
  answering. We reserve the right to give 0 points to submissions that
  fail to do this.
\end{itemize}

\pagenumbering{gobble}

\begin{enumerate}
\def\labelenumi{\arabic{enumi}.}
\item
  (15 points) Write a \textbf{definition} for a void function called
  \texttt{check\_it()} that takes 3 integer arguments and prints ``YES''
  if the arguments are in ascending order. Otherwise, it returns ``NO''.

  For example, \texttt{check\_it(1,\ 2,\ 6)} returns ``YES'', but
  \texttt{check\_it(6,\ 6,\ 1)} returns ``NO''.
\end{enumerate}

\begin{verbatim}



\end{verbatim}

\pagebreak

\begin{enumerate}
\def\labelenumi{\arabic{enumi}.}
\setcounter{enumi}{1}
\item
  (15 points) Write a \textbf{definition} for a void function called
  \texttt{roll()} that takes no arguments and prints a \emph{random
  integer number} between 2 and 13 (inclusive) every time the function
  is called.

  For example, if, in a program, you do this:

\begin{Shaded}
\begin{Highlighting}[]
\ControlFlowTok{for}\NormalTok{ (}\DataTypeTok{int}\NormalTok{ k = }\DecValTok{0}\NormalTok{; k \textless{} }\DecValTok{5}\NormalTok{; k++) \{}
\NormalTok{    roll();}
\NormalTok{\}}
\end{Highlighting}
\end{Shaded}

  You could get an output like this (note the newlines after each
  number). Assume your program already has included \texttt{cstdlib} and
  \texttt{ctime} libraries and has seeded the random number generator in
  the \texttt{main()} function. Just give the function definition for
  the answer.

\begin{verbatim}
4
2
11
5
7
\end{verbatim}
\end{enumerate}

\begin{verbatim}

\end{verbatim}

\pagebreak

\begin{enumerate}
\def\labelenumi{\arabic{enumi}.}
\setcounter{enumi}{2}
\tightlist
\item
  (10 points) Consider the follow code snippet:
\end{enumerate}

\begin{Shaded}
\begin{Highlighting}[]
\DataTypeTok{int}\NormalTok{ x = }\DecValTok{10}\NormalTok{;}
\ControlFlowTok{while}\NormalTok{ (x{-}{-} \textgreater{}= }\DecValTok{3}\NormalTok{) \{}
\NormalTok{    cout \textless{}\textless{} x \textless{}\textless{} }\StringTok{" "}\NormalTok{;}
    \ControlFlowTok{if}\NormalTok{ (!(x \% }\DecValTok{3}\NormalTok{)) \{}
\NormalTok{        cout \textless{}\textless{} }\StringTok{"Buzz! "}\NormalTok{;}
        \ControlFlowTok{if}\NormalTok{ ((x \% }\DecValTok{2}\NormalTok{) == }\DecValTok{0}\NormalTok{) \{}
\NormalTok{            cout \textless{}\textless{} }\StringTok{"Fizz!"}\NormalTok{;}
\NormalTok{        \}}
\NormalTok{    \}}
    \ControlFlowTok{else}\NormalTok{ \{}
\NormalTok{        cout \textless{}\textless{} }\StringTok{"..."}\NormalTok{ \textless{}\textless{} endl;}
\NormalTok{    \}}
\NormalTok{\}}
\end{Highlighting}
\end{Shaded}

\begin{enumerate}
\def\labelenumi{\alph{enumi}.}
\tightlist
\item
  (3 points) Write what this code will print out exactly.
\end{enumerate}

\begin{verbatim}


\end{verbatim}

\begin{enumerate}
\def\labelenumi{\alph{enumi}.}
\setcounter{enumi}{1}
\tightlist
\item
  (7 points) Explain step-by-step \textbf{why} the program prints out
  what it does.
\end{enumerate}

\begin{verbatim}


\end{verbatim}

\pagebreak

\begin{enumerate}
\def\labelenumi{\arabic{enumi}.}
\setcounter{enumi}{3}
\tightlist
\item
  (5 points) Explain the difference between these 2 snippets of code.
\end{enumerate}

\begin{Shaded}
\begin{Highlighting}[]
\ControlFlowTok{for}\NormalTok{ (}\DataTypeTok{int}\NormalTok{ i = }\DecValTok{0}\NormalTok{; i \textless{} }\DecValTok{10}\NormalTok{; i++) \{}
\NormalTok{    cout \textless{}\textless{} i;}
\NormalTok{\}}
\end{Highlighting}
\end{Shaded}

\texttt{-\/-\/-\/-\/-\/-\/-\/-\/-\/-\/-\/-\/-\/-\/-\/-\/-\/-\/-\/-\/-\/-\/-\/-\/-\/-\/-\/-\/-\/-}

\begin{Shaded}
\begin{Highlighting}[]
\DataTypeTok{int}\NormalTok{ i;}
\ControlFlowTok{for}\NormalTok{ (i = }\DecValTok{0}\NormalTok{; i \textless{} }\DecValTok{10}\NormalTok{; i++) \{}
\NormalTok{    cout \textless{}\textless{} i;}
\NormalTok{\}}
\end{Highlighting}
\end{Shaded}

\begin{verbatim}








\end{verbatim}

\begin{enumerate}
\def\labelenumi{\arabic{enumi}.}
\setcounter{enumi}{4}
\tightlist
\item
  (6 points) Consider the code below.
\end{enumerate}

\begin{Shaded}
\begin{Highlighting}[]
\DataTypeTok{int}\NormalTok{ a = }\DecValTok{7}\NormalTok{, b = }\DecValTok{9}\NormalTok{;}
\NormalTok{cout \textless{}\textless{} }\StringTok{"Here is "}\NormalTok{;}
\ControlFlowTok{while}\NormalTok{ (a++ \% b != }\DecValTok{0}\NormalTok{) \{}
\NormalTok{    cout \textless{}\textless{} a \textless{}\textless{} }\StringTok{" "}\NormalTok{;}
\NormalTok{    b += }\DecValTok{2}\NormalTok{;}
\NormalTok{    a {-}= }\DecValTok{2}\NormalTok{;}
\NormalTok{\}}
\NormalTok{cout \textless{}\textless{} endl;}
\end{Highlighting}
\end{Shaded}

\begin{enumerate}
\def\labelenumi{\alph{enumi}.}
\tightlist
\item
  (2 points) Write what this code will print exactly.
\end{enumerate}

\begin{verbatim}


\end{verbatim}

\begin{enumerate}
\def\labelenumi{\alph{enumi}.}
\setcounter{enumi}{1}
\tightlist
\item
  (4 points) How is the value of variable \texttt{a} changing? Show your
  work!
\end{enumerate}

\begin{verbatim}





\end{verbatim}

\begin{enumerate}
\def\labelenumi{\arabic{enumi}.}
\setcounter{enumi}{5}
\tightlist
\item
  (15 points) Consider the following \texttt{main()} function.
\end{enumerate}

\begin{Shaded}
\begin{Highlighting}[]
\DataTypeTok{int}\NormalTok{ main() \{}
    \DataTypeTok{int}\NormalTok{ x = }\DecValTok{10}\NormalTok{, y = }\DecValTok{20}\NormalTok{, z = }\DecValTok{30}\NormalTok{;}
\NormalTok{    shift(x, y, z);}
\NormalTok{    cout \textless{}\textless{} x \textless{}\textless{} }\StringTok{" "}\NormalTok{ \textless{}\textless{} y \textless{}\textless{} }\StringTok{" "}\NormalTok{ \textless{}\textless{} z \textless{}\textless{} endl;}
    \ControlFlowTok{return} \DecValTok{0}\NormalTok{;}
\NormalTok{\}}
\end{Highlighting}
\end{Shaded}

The \textbf{body} of the \texttt{shift()} function is as follows.

\begin{Shaded}
\begin{Highlighting}[]
\NormalTok{\{}
    \DataTypeTok{int}\NormalTok{ temp;}
\NormalTok{    temp = var1;}
\NormalTok{    var1 = var2;}
\NormalTok{    var2 = var3;}
\NormalTok{    var3 = temp;}
\NormalTok{\}}
\end{Highlighting}
\end{Shaded}

What will this program print for each of the following function
declarations for \texttt{shift()}. \textbf{Explain why!}

\begin{enumerate}
\def\labelenumi{\alph{enumi}.}
\tightlist
\item
  (5 points) \texttt{void\ shift(int\ var1,\ int\ var2,\ int\ \&var3);}
\end{enumerate}

\begin{verbatim}




\end{verbatim}

\begin{enumerate}
\def\labelenumi{\alph{enumi}.}
\setcounter{enumi}{1}
\tightlist
\item
  (5 points)
  \texttt{void\ shift(int\ \&var1,\ int\ \&var2,\ int\ var3);}
\end{enumerate}

\begin{verbatim}




\end{verbatim}

\begin{enumerate}
\def\labelenumi{\alph{enumi}.}
\setcounter{enumi}{2}
\tightlist
\item
  (5 points)
  \texttt{void\ shift(int\ \&var1,\ int\ \&var2,\ int\ \&var3);}
\end{enumerate}

\begin{verbatim}




\end{verbatim}

\begin{enumerate}
\def\labelenumi{\arabic{enumi}.}
\setcounter{enumi}{6}
\tightlist
\item
  (6 points) We talked about 3 concepts related to programmer-defined
  functions: (1) function declaration, (2) function definition, and (3)
  function call.
\end{enumerate}

\begin{Shaded}
\begin{Highlighting}[]
\DecValTok{1}  \ErrorTok{\#include \textless{}iostream\textgreater{}}
\DecValTok{2}  \KeywordTok{using} \KeywordTok{namespace}\NormalTok{ std;}
\DecValTok{3}  \DataTypeTok{bool}\NormalTok{ isDivisibleBy(}\DataTypeTok{int}\NormalTok{ a, }\DataTypeTok{int}\NormalTok{ b);}
\DecValTok{4}
\DecValTok{5}  \DataTypeTok{int}\NormalTok{ main() \{}
\DecValTok{6}\NormalTok{      cout \textless{}\textless{} }\StringTok{"15 divisible by 5? "}\NormalTok{ \textless{}\textless{} isDivisibleBy(}\DecValTok{15}\NormalTok{, }\DecValTok{5}\NormalTok{) \textless{}\textless{} endl;}
\DecValTok{7}      \ControlFlowTok{return} \DecValTok{0}\NormalTok{;}
\DecValTok{8}\NormalTok{  \}}
\DecValTok{9}
\DecValTok{10} \DataTypeTok{bool}\NormalTok{ isDivisibleBy(}\DataTypeTok{int}\NormalTok{ a, }\DataTypeTok{int}\NormalTok{ b) \{}
\DecValTok{11}     \ControlFlowTok{return}\NormalTok{ (a \% b == }\DecValTok{0}\NormalTok{);}
\DecValTok{12}\NormalTok{ \}}
\end{Highlighting}
\end{Shaded}

\begin{enumerate}
\def\labelenumi{\alph{enumi}.}
\tightlist
\item
  (2 points) List the line number(s) for the function declaration of
  \texttt{isDivisibleBy()}.
\end{enumerate}

\begin{verbatim}
\end{verbatim}

\begin{enumerate}
\def\labelenumi{\alph{enumi}.}
\setcounter{enumi}{1}
\tightlist
\item
  (2 points) List the line number(s) for the function definition of
  \texttt{isDivisibleBy()}.
\end{enumerate}

\begin{verbatim}
\end{verbatim}

\begin{enumerate}
\def\labelenumi{\alph{enumi}.}
\setcounter{enumi}{2}
\tightlist
\item
  (2 points) List the line number(s) for the function calls of
  \texttt{isDivisibleBy()}.
\end{enumerate}

\begin{verbatim}
\end{verbatim}

\begin{enumerate}
\def\labelenumi{\arabic{enumi}.}
\setcounter{enumi}{7}
\tightlist
\item
  (3 points) What is a flag in a program and what use is it? (Read
  Chapter 3.)
\end{enumerate}

\begin{verbatim}




\end{verbatim}

\end{document}
