% Options for packages loaded elsewhere
\PassOptionsToPackage{unicode}{hyperref}
\PassOptionsToPackage{hyphens}{url}
%
\documentclass[
]{article}
\usepackage{amsmath,amssymb}
\usepackage{lmodern}
\usepackage{ifxetex,ifluatex}
\ifnum 0\ifxetex 1\fi\ifluatex 1\fi=0 % if pdftex
  \usepackage[T1]{fontenc}
  \usepackage[utf8]{inputenc}
  \usepackage{textcomp} % provide euro and other symbols
\else % if luatex or xetex
  \usepackage{unicode-math}
  \defaultfontfeatures{Scale=MatchLowercase}
  \defaultfontfeatures[\rmfamily]{Ligatures=TeX,Scale=1}
\fi
% Use upquote if available, for straight quotes in verbatim environments
\IfFileExists{upquote.sty}{\usepackage{upquote}}{}
\IfFileExists{microtype.sty}{% use microtype if available
  \usepackage[]{microtype}
  \UseMicrotypeSet[protrusion]{basicmath} % disable protrusion for tt fonts
}{}
\makeatletter
\@ifundefined{KOMAClassName}{% if non-KOMA class
  \IfFileExists{parskip.sty}{%
    \usepackage{parskip}
  }{% else
    \setlength{\parindent}{0pt}
    \setlength{\parskip}{6pt plus 2pt minus 1pt}}
}{% if KOMA class
  \KOMAoptions{parskip=half}}
\makeatother
\usepackage{xcolor}
\IfFileExists{xurl.sty}{\usepackage{xurl}}{} % add URL line breaks if available
\IfFileExists{bookmark.sty}{\usepackage{bookmark}}{\usepackage{hyperref}}
\hypersetup{
  pdftitle={Homework 7: Structs and Classes},
  pdfauthor={CS16 - Winter 2021},
  hidelinks,
  pdfcreator={LaTeX via pandoc}}
\urlstyle{same} % disable monospaced font for URLs
\usepackage{color}
\usepackage{fancyvrb}
\newcommand{\VerbBar}{|}
\newcommand{\VERB}{\Verb[commandchars=\\\{\}]}
\DefineVerbatimEnvironment{Highlighting}{Verbatim}{commandchars=\\\{\}}
% Add ',fontsize=\small' for more characters per line
\newenvironment{Shaded}{}{}
\newcommand{\AlertTok}[1]{\textcolor[rgb]{1.00,0.00,0.00}{\textbf{#1}}}
\newcommand{\AnnotationTok}[1]{\textcolor[rgb]{0.38,0.63,0.69}{\textbf{\textit{#1}}}}
\newcommand{\AttributeTok}[1]{\textcolor[rgb]{0.49,0.56,0.16}{#1}}
\newcommand{\BaseNTok}[1]{\textcolor[rgb]{0.25,0.63,0.44}{#1}}
\newcommand{\BuiltInTok}[1]{#1}
\newcommand{\CharTok}[1]{\textcolor[rgb]{0.25,0.44,0.63}{#1}}
\newcommand{\CommentTok}[1]{\textcolor[rgb]{0.38,0.63,0.69}{\textit{#1}}}
\newcommand{\CommentVarTok}[1]{\textcolor[rgb]{0.38,0.63,0.69}{\textbf{\textit{#1}}}}
\newcommand{\ConstantTok}[1]{\textcolor[rgb]{0.53,0.00,0.00}{#1}}
\newcommand{\ControlFlowTok}[1]{\textcolor[rgb]{0.00,0.44,0.13}{\textbf{#1}}}
\newcommand{\DataTypeTok}[1]{\textcolor[rgb]{0.56,0.13,0.00}{#1}}
\newcommand{\DecValTok}[1]{\textcolor[rgb]{0.25,0.63,0.44}{#1}}
\newcommand{\DocumentationTok}[1]{\textcolor[rgb]{0.73,0.13,0.13}{\textit{#1}}}
\newcommand{\ErrorTok}[1]{\textcolor[rgb]{1.00,0.00,0.00}{\textbf{#1}}}
\newcommand{\ExtensionTok}[1]{#1}
\newcommand{\FloatTok}[1]{\textcolor[rgb]{0.25,0.63,0.44}{#1}}
\newcommand{\FunctionTok}[1]{\textcolor[rgb]{0.02,0.16,0.49}{#1}}
\newcommand{\ImportTok}[1]{#1}
\newcommand{\InformationTok}[1]{\textcolor[rgb]{0.38,0.63,0.69}{\textbf{\textit{#1}}}}
\newcommand{\KeywordTok}[1]{\textcolor[rgb]{0.00,0.44,0.13}{\textbf{#1}}}
\newcommand{\NormalTok}[1]{#1}
\newcommand{\OperatorTok}[1]{\textcolor[rgb]{0.40,0.40,0.40}{#1}}
\newcommand{\OtherTok}[1]{\textcolor[rgb]{0.00,0.44,0.13}{#1}}
\newcommand{\PreprocessorTok}[1]{\textcolor[rgb]{0.74,0.48,0.00}{#1}}
\newcommand{\RegionMarkerTok}[1]{#1}
\newcommand{\SpecialCharTok}[1]{\textcolor[rgb]{0.25,0.44,0.63}{#1}}
\newcommand{\SpecialStringTok}[1]{\textcolor[rgb]{0.73,0.40,0.53}{#1}}
\newcommand{\StringTok}[1]{\textcolor[rgb]{0.25,0.44,0.63}{#1}}
\newcommand{\VariableTok}[1]{\textcolor[rgb]{0.10,0.09,0.49}{#1}}
\newcommand{\VerbatimStringTok}[1]{\textcolor[rgb]{0.25,0.44,0.63}{#1}}
\newcommand{\WarningTok}[1]{\textcolor[rgb]{0.38,0.63,0.69}{\textbf{\textit{#1}}}}
\usepackage{longtable,booktabs,array}
\usepackage{calc} % for calculating minipage widths
% Correct order of tables after \paragraph or \subparagraph
\usepackage{etoolbox}
\makeatletter
\patchcmd\longtable{\par}{\if@noskipsec\mbox{}\fi\par}{}{}
\makeatother
% Allow footnotes in longtable head/foot
\IfFileExists{footnotehyper.sty}{\usepackage{footnotehyper}}{\usepackage{footnote}}
\makesavenoteenv{longtable}
\setlength{\emergencystretch}{3em} % prevent overfull lines
\providecommand{\tightlist}{%
  \setlength{\itemsep}{0pt}\setlength{\parskip}{0pt}}
\setcounter{secnumdepth}{-\maxdimen} % remove section numbering
\ifluatex
  \usepackage{selnolig}  % disable illegal ligatures
\fi

\title{Homework 7: Structs and Classes}
\author{CS16 - Winter 2021}
\date{}

\begin{document}
\maketitle

\begin{longtable}[]{@{}cl@{}}
\toprule
\endhead
\textbf{Due:} & Thursday, February 25, 2021 (11:59 PM
PST) \\ \addlinespace
\textbf{Points:} & 50 \\ \addlinespace
\textbf{Name:} &
\texttt{\_\_\_\_\_\_\_\_\_\_\_\_\_\_\_\_\_\_\_\_\_\_\_\_\_\_\_\_\_\_\_\_\_\_\_\_\_\_\_\_\_\_\_\_\_\_\_\_\_\_\_\_\_\_\_} \\ \addlinespace
\textbf{Homework buddy:} &
\texttt{\_\_\_\_\_\_\_\_\_\_\_\_\_\_\_\_\_\_\_\_\_\_\_\_\_\_\_\_\_\_\_\_\_\_\_\_\_\_\_\_\_\_\_\_\_\_\_\_\_\_\_\_\_\_\_} \\ \addlinespace
\bottomrule
\end{longtable}

\begin{itemize}
\tightlist
\item
  You may collaborate on this homework with \textbf{at most} one person,
  an optional ``homework buddy.''
\item
  \textbf{Submission instructions:} All questions are to be written
  (either by hand or typed) \emph{in the provided spaces} and turned in
  as a single PDF on Gradescope. If you submit handwritten solutions
  write legibly. We reserve the right to give 0 points to answers we
  cannot read. When you submit your answer on Gradescope, \textbf{be
  sure to select which portions of your answer correspond to which
  problem} and clearly mark on the page itself which problem you are
  answering. We reserve the right to give 0 points to submissions that
  fail to do this.
\end{itemize}

\pagenumbering{gobble}

\begin{enumerate}
\def\labelenumi{\arabic{enumi}.}
\tightlist
\item
  (4 points) Write a definition for a structure type for records
  consisting of a person's wage rate (dollars per hour), accrued
  vacation (in whole days), and status (hourly or salaried represented
  as either `H' or `S', respectively). Call the type
  \texttt{EmployeeRecord}.
\end{enumerate}

\begin{verbatim}
\end{verbatim}

\pagebreak

\begin{enumerate}
\def\labelenumi{\arabic{enumi}.}
\setcounter{enumi}{1}
\tightlist
\item
  (6 points) Given the following structures defined:
\end{enumerate}

\begin{Shaded}
\begin{Highlighting}[]
\KeywordTok{struct}\NormalTok{ Date \{}
   \DataTypeTok{int}\NormalTok{ day;}
   \DataTypeTok{int}\NormalTok{ month;}
   \DataTypeTok{int}\NormalTok{ year;}
\NormalTok{\};}

\KeywordTok{struct}\NormalTok{ Person \{}
\NormalTok{   string name;}
\NormalTok{   Date dateOfBirth;}
\NormalTok{\};}

\KeywordTok{struct}\NormalTok{ ProjectTeam \{}
\NormalTok{   Person MemberA, MemberB;}
\NormalTok{   Person Leader;}
\NormalTok{   string projectName;}
   \DataTypeTok{double}\NormalTok{ projectBudget;}
\NormalTok{   Date projectDueDate;}
\NormalTok{\};}
\end{Highlighting}
\end{Shaded}

If we declare \texttt{ProjectTeam\ TheATeam;} which was then initialized
fully and correctly:

\begin{enumerate}
\def\labelenumi{\alph{enumi}.}
\tightlist
\item
  (2 points) How would you print (to standard out) the project budget
  for \texttt{TheATeam}?
\end{enumerate}

\begin{verbatim}
\end{verbatim}

\begin{enumerate}
\def\labelenumi{\alph{enumi}.}
\setcounter{enumi}{1}
\tightlist
\item
  (2 points) How would you print (to standard out) the name of Member B
  of \texttt{TheATeam}?
\end{enumerate}

\begin{verbatim}
\end{verbatim}

\begin{enumerate}
\def\labelenumi{\alph{enumi}.}
\setcounter{enumi}{2}
\tightlist
\item
  (2 points) How would you print (to standard out) the year that the
  project leader of \texttt{TheATeam} was born?
\end{enumerate}

\begin{verbatim}
\end{verbatim}

\pagebreak

\begin{enumerate}
\def\labelenumi{\arabic{enumi}.}
\setcounter{enumi}{2}
\tightlist
\item
  (5 points) What's the difference between a \texttt{struct} and
  \texttt{class} in C++?
\end{enumerate}

\begin{verbatim}









\end{verbatim}

\begin{enumerate}
\def\labelenumi{\arabic{enumi}.}
\setcounter{enumi}{3}
\tightlist
\item
  (5 points) What's the difference between \texttt{public} and
  \texttt{private} members of a class in C++?
\end{enumerate}

\begin{verbatim}









\end{verbatim}

\begin{enumerate}
\def\labelenumi{\arabic{enumi}.}
\setcounter{enumi}{4}
\tightlist
\item
  (5 points) What are class constructors?
\end{enumerate}

\begin{verbatim}









\end{verbatim}

\pagebreak

\begin{enumerate}
\def\labelenumi{\arabic{enumi}.}
\setcounter{enumi}{5}
\tightlist
\item
  (25 points) Suppose your program contains the following class
  definition:
\end{enumerate}

\begin{Shaded}
\begin{Highlighting}[]
\KeywordTok{class}\NormalTok{ Point \{}
   \KeywordTok{public}\NormalTok{:}
\NormalTok{      Point(}\DataTypeTok{double}\NormalTok{ n1, }\DataTypeTok{double}\NormalTok{ n2);}
\NormalTok{      Point(); }\CommentTok{// initializes member variables to 0}
      \DataTypeTok{double}\NormalTok{ get\_x(); }\CommentTok{// returns value of x}
      \DataTypeTok{double}\NormalTok{ get\_y(); }\CommentTok{// returns value of y}
      \DataTypeTok{void}\NormalTok{ set\_x(}\DataTypeTok{double}\NormalTok{ n); }\CommentTok{// sets a new value for x}
      \DataTypeTok{void}\NormalTok{ set\_y(}\DataTypeTok{double}\NormalTok{ n); }\CommentTok{// sets a new value for y}
   \KeywordTok{private}\NormalTok{:}
      \DataTypeTok{double}\NormalTok{ x, y;}
\NormalTok{\};}
\end{Highlighting}
\end{Shaded}

\begin{enumerate}
\def\labelenumi{\alph{enumi}.}
\tightlist
\item
  (12 points) Given the comments shown, give definitions to all 6 of
  these member functions/constructors:
\end{enumerate}

\begin{verbatim}
\end{verbatim}

\pagebreak

For points \((x_1, y_1), (x_2, y_2)\), the Euclidean distance formula is
given by:

\begin{equation*}
d = \sqrt{(x_2 - x_1)^2 + (y_2 - y_1)^2}
\end{equation*}

Suppose we want to add a member function to the \texttt{Point} class
that computes the distance between a given point and itself. Call it
\texttt{distanceFrom}. The function should take as argument
\emph{another} object of type \texttt{Point} and return the computed
distance. Assume that the \texttt{\textless{}cmath\textgreater{}}
library is already included.

\begin{enumerate}
\def\labelenumi{\alph{enumi}.}
\setcounter{enumi}{1}
\tightlist
\item
  (2 points) Give the member function \emph{declaration} for the
  \texttt{distanceFrom} member function.
\end{enumerate}

\begin{verbatim}
\end{verbatim}

\begin{enumerate}
\def\labelenumi{\alph{enumi}.}
\setcounter{enumi}{2}
\tightlist
\item
  (4 points) Give the member function \emph{definition} for
  \texttt{distanceFrom}.
\end{enumerate}

\begin{verbatim}






\end{verbatim}

For a point \((x, y)\), we can rotate it by \(\theta\) degrees to obtain
a new point \((x', y')\):

\begin{equation*}
x' = x\, cos(\theta) - y\, sin(\theta)
\end{equation*} \begin{equation*}
y' = x\, sin(\theta) + y\, cos(\theta)
\end{equation*}

Suppose we want to add a member function to the \texttt{Point} class
that rotates the point by a given degree and \emph{updates} the values
for the member variables \texttt{x} and \texttt{y}. Call it
\texttt{rotate}. The function should take as argument a double
representing the degree \(\theta\). Assume that the
\texttt{\textless{}cmath\textgreater{}} library is already included.

\begin{enumerate}
\def\labelenumi{\alph{enumi}.}
\setcounter{enumi}{3}
\tightlist
\item
  (2 points) Give the member function \emph{declaration} for the
  \texttt{rotate} member function.
\end{enumerate}

\begin{verbatim}
\end{verbatim}

\begin{enumerate}
\def\labelenumi{\alph{enumi}.}
\setcounter{enumi}{4}
\tightlist
\item
  (5 points) Give the member function \emph{definition} for
  \texttt{rotate}.
\end{enumerate}

\begin{verbatim}






\end{verbatim}

\end{document}
