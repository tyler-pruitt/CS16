% Options for packages loaded elsewhere
\PassOptionsToPackage{unicode}{hyperref}
\PassOptionsToPackage{hyphens}{url}
%
\documentclass[
]{article}
\usepackage{lmodern}
\usepackage{amssymb,amsmath}
\usepackage{ifxetex,ifluatex}
\ifnum 0\ifxetex 1\fi\ifluatex 1\fi=0 % if pdftex
  \usepackage[T1]{fontenc}
  \usepackage[utf8]{inputenc}
  \usepackage{textcomp} % provide euro and other symbols
\else % if luatex or xetex
  \usepackage{unicode-math}
  \defaultfontfeatures{Scale=MatchLowercase}
  \defaultfontfeatures[\rmfamily]{Ligatures=TeX,Scale=1}
\fi
% Use upquote if available, for straight quotes in verbatim environments
\IfFileExists{upquote.sty}{\usepackage{upquote}}{}
\IfFileExists{microtype.sty}{% use microtype if available
  \usepackage[]{microtype}
  \UseMicrotypeSet[protrusion]{basicmath} % disable protrusion for tt fonts
}{}
\makeatletter
\@ifundefined{KOMAClassName}{% if non-KOMA class
  \IfFileExists{parskip.sty}{%
    \usepackage{parskip}
  }{% else
    \setlength{\parindent}{0pt}
    \setlength{\parskip}{6pt plus 2pt minus 1pt}}
}{% if KOMA class
  \KOMAoptions{parskip=half}}
\makeatother
\usepackage{xcolor}
\IfFileExists{xurl.sty}{\usepackage{xurl}}{} % add URL line breaks if available
\IfFileExists{bookmark.sty}{\usepackage{bookmark}}{\usepackage{hyperref}}
\hypersetup{
  pdftitle={Homework 5: Strings},
  pdfauthor={CS16 - Winter 2021},
  hidelinks,
  pdfcreator={LaTeX via pandoc}}
\urlstyle{same} % disable monospaced font for URLs
\usepackage{color}
\usepackage{fancyvrb}
\newcommand{\VerbBar}{|}
\newcommand{\VERB}{\Verb[commandchars=\\\{\}]}
\DefineVerbatimEnvironment{Highlighting}{Verbatim}{commandchars=\\\{\}}
% Add ',fontsize=\small' for more characters per line
\newenvironment{Shaded}{}{}
\newcommand{\AlertTok}[1]{\textcolor[rgb]{1.00,0.00,0.00}{\textbf{#1}}}
\newcommand{\AnnotationTok}[1]{\textcolor[rgb]{0.38,0.63,0.69}{\textbf{\textit{#1}}}}
\newcommand{\AttributeTok}[1]{\textcolor[rgb]{0.49,0.56,0.16}{#1}}
\newcommand{\BaseNTok}[1]{\textcolor[rgb]{0.25,0.63,0.44}{#1}}
\newcommand{\BuiltInTok}[1]{#1}
\newcommand{\CharTok}[1]{\textcolor[rgb]{0.25,0.44,0.63}{#1}}
\newcommand{\CommentTok}[1]{\textcolor[rgb]{0.38,0.63,0.69}{\textit{#1}}}
\newcommand{\CommentVarTok}[1]{\textcolor[rgb]{0.38,0.63,0.69}{\textbf{\textit{#1}}}}
\newcommand{\ConstantTok}[1]{\textcolor[rgb]{0.53,0.00,0.00}{#1}}
\newcommand{\ControlFlowTok}[1]{\textcolor[rgb]{0.00,0.44,0.13}{\textbf{#1}}}
\newcommand{\DataTypeTok}[1]{\textcolor[rgb]{0.56,0.13,0.00}{#1}}
\newcommand{\DecValTok}[1]{\textcolor[rgb]{0.25,0.63,0.44}{#1}}
\newcommand{\DocumentationTok}[1]{\textcolor[rgb]{0.73,0.13,0.13}{\textit{#1}}}
\newcommand{\ErrorTok}[1]{\textcolor[rgb]{1.00,0.00,0.00}{\textbf{#1}}}
\newcommand{\ExtensionTok}[1]{#1}
\newcommand{\FloatTok}[1]{\textcolor[rgb]{0.25,0.63,0.44}{#1}}
\newcommand{\FunctionTok}[1]{\textcolor[rgb]{0.02,0.16,0.49}{#1}}
\newcommand{\ImportTok}[1]{#1}
\newcommand{\InformationTok}[1]{\textcolor[rgb]{0.38,0.63,0.69}{\textbf{\textit{#1}}}}
\newcommand{\KeywordTok}[1]{\textcolor[rgb]{0.00,0.44,0.13}{\textbf{#1}}}
\newcommand{\NormalTok}[1]{#1}
\newcommand{\OperatorTok}[1]{\textcolor[rgb]{0.40,0.40,0.40}{#1}}
\newcommand{\OtherTok}[1]{\textcolor[rgb]{0.00,0.44,0.13}{#1}}
\newcommand{\PreprocessorTok}[1]{\textcolor[rgb]{0.74,0.48,0.00}{#1}}
\newcommand{\RegionMarkerTok}[1]{#1}
\newcommand{\SpecialCharTok}[1]{\textcolor[rgb]{0.25,0.44,0.63}{#1}}
\newcommand{\SpecialStringTok}[1]{\textcolor[rgb]{0.73,0.40,0.53}{#1}}
\newcommand{\StringTok}[1]{\textcolor[rgb]{0.25,0.44,0.63}{#1}}
\newcommand{\VariableTok}[1]{\textcolor[rgb]{0.10,0.09,0.49}{#1}}
\newcommand{\VerbatimStringTok}[1]{\textcolor[rgb]{0.25,0.44,0.63}{#1}}
\newcommand{\WarningTok}[1]{\textcolor[rgb]{0.38,0.63,0.69}{\textbf{\textit{#1}}}}
\usepackage{longtable,booktabs}
% Correct order of tables after \paragraph or \subparagraph
\usepackage{etoolbox}
\makeatletter
\patchcmd\longtable{\par}{\if@noskipsec\mbox{}\fi\par}{}{}
\makeatother
% Allow footnotes in longtable head/foot
\IfFileExists{footnotehyper.sty}{\usepackage{footnotehyper}}{\usepackage{footnote}}
\makesavenoteenv{longtable}
\setlength{\emergencystretch}{3em} % prevent overfull lines
\providecommand{\tightlist}{%
  \setlength{\itemsep}{0pt}\setlength{\parskip}{0pt}}
\setcounter{secnumdepth}{-\maxdimen} % remove section numbering
\ifluatex
  \usepackage{selnolig}  % disable illegal ligatures
\fi

\title{Homework 5: Strings}
\author{CS16 - Winter 2021}
\date{}

\begin{document}
\maketitle

\begin{longtable}[]{@{}cl@{}}
\toprule
\endhead
\textbf{Due:} & Thursday, February 11, 2021 (11:59 PM
PST)\tabularnewline
\textbf{Points:} & 100\tabularnewline
\textbf{Name:} &
\texttt{\_\_\_\_\_\_\_\_\_\_\_\_\_\_\_\_\_\_\_\_\_\_\_\_\_\_\_\_\_\_\_\_\_\_\_\_\_\_\_\_\_\_\_\_\_\_\_\_\_\_\_\_\_\_\_}\tabularnewline
\textbf{Homework buddy:} &
\texttt{\_\_\_\_\_\_\_\_\_\_\_\_\_\_\_\_\_\_\_\_\_\_\_\_\_\_\_\_\_\_\_\_\_\_\_\_\_\_\_\_\_\_\_\_\_\_\_\_\_\_\_\_\_\_\_}\tabularnewline
\bottomrule
\end{longtable}

\begin{itemize}
\tightlist
\item
  You may collaborate on this homework with \textbf{at most} one person,
  an optional ``homework buddy.''
\item
  \textbf{Submission instructions:} All questions are to be written
  (either by hand or typed) \emph{in the provided spaces} and turned in
  as a single PDF on Gradescope. If you submit handwritten solutions
  write legibly. We reserve the right to give 0 points to answers we
  cannot read. When you submit your answer on Gradescope, \textbf{be
  sure to select which portions of your answer correspond to which
  problem} and clearly mark on the page itself which problem you are
  answering. We reserve the right to give 0 points to submissions that
  fail to do this.
\end{itemize}

\pagenumbering{gobble}

\begin{enumerate}
\def\labelenumi{\arabic{enumi}.}
\tightlist
\item
  (10 points) Which of these are correct usage (syntax) of a single
  statement on a string variable called \texttt{message}, and which of
  these are incorrect usage (and \emph{very briefly} \textbf{why}).
  Variables \texttt{n} and \texttt{m} are \texttt{int} types.
\end{enumerate}

\begin{enumerate}
\def\labelenumi{\alph{enumi}.}
\tightlist
\item
  (2 points) \texttt{message.erase(n,\ m);}
\end{enumerate}

\begin{verbatim}

\end{verbatim}

\begin{enumerate}
\def\labelenumi{\alph{enumi}.}
\setcounter{enumi}{1}
\tightlist
\item
  (2 points) \texttt{message\ =\ message.erase(n,\ m);}
\end{enumerate}

\begin{verbatim}

\end{verbatim}

\begin{enumerate}
\def\labelenumi{\alph{enumi}.}
\setcounter{enumi}{2}
\tightlist
\item
  (2 points) \texttt{cout\ \textless{}\textless{}\ message.find(n);}
\end{enumerate}

\begin{verbatim}

\end{verbatim}

\begin{enumerate}
\def\labelenumi{\alph{enumi}.}
\setcounter{enumi}{3}
\tightlist
\item
  (2 points) \texttt{message.size()\ =\ n;}
\end{enumerate}

\begin{verbatim}

\end{verbatim}

\begin{enumerate}
\def\labelenumi{\alph{enumi}.}
\setcounter{enumi}{4}
\tightlist
\item
  (2 points) \texttt{cout\ \textless{}\textless{}\ message.rfind("x");}
\end{enumerate}

\begin{verbatim}

\end{verbatim}

\begin{enumerate}
\def\labelenumi{\arabic{enumi}.}
\setcounter{enumi}{1}
\tightlist
\item
  (10 points) The following code takes in a string input from the user
  and performs an integer multiplication, as seen in the example run
  here. Note that the input string will contain the asterisk character
  \texttt{\textquotesingle{}*\textquotesingle{}}:
\end{enumerate}

\begin{verbatim}
Enter 2 integer numbers to be multiplied, like this: num1*num2: 15*3
The answer is: 45
\end{verbatim}

Complete the missing code below that performs this task (it can be done
in 2 lines, but you can use more if you like).

\begin{Shaded}
\begin{Highlighting}[]
\NormalTok{string s; }\DataTypeTok{int}\NormalTok{ k(}\DecValTok{0}\NormalTok{);}
\NormalTok{cout \textless{}\textless{} }\StringTok{"Enter 2 integer numbers to be multiplied, like this: num1*num2: "}\NormalTok{;}
\NormalTok{cin \textgreater{}\textgreater{} s;}




\NormalTok{cout \textless{}\textless{} }\StringTok{"The answer is: "}\NormalTok{ \textless{}\textless{} k \textless{}\textless{} endl; }
\end{Highlighting}
\end{Shaded}

\begin{enumerate}
\def\labelenumi{\arabic{enumi}.}
\setcounter{enumi}{2}
\tightlist
\item
  (20 points) Given the declaration of a C-string variable, where
  \texttt{MAX} is a defined constant: \texttt{char\ buffer{[}MAX{]};}
\end{enumerate}

The C-string variable \texttt{buffer} has previously been assigned in
code not shown here. For correct C- string variables, the following loop
reassigns all positions of \texttt{buffer} the value `z', leaving the
length the same as before. Assume this code fragment is embedded in an
otherwise complete and correct program. Answer the questions following
this code fragment:

\begin{Shaded}
\begin{Highlighting}[]
\DataTypeTok{int}\NormalTok{ index = }\DecValTok{0}\NormalTok{;}
\ControlFlowTok{while}\NormalTok{ (buffer[index] != }\CharTok{\textquotesingle{}}\SpecialCharTok{\textbackslash{}0}\CharTok{\textquotesingle{}}\NormalTok{) \{}
\NormalTok{   buffer[index] = }\CharTok{\textquotesingle{}z\textquotesingle{}}\NormalTok{;}
\NormalTok{   index++;}
\NormalTok{\}}
\end{Highlighting}
\end{Shaded}

\begin{enumerate}
\def\labelenumi{\alph{enumi}.}
\tightlist
\item
  (10 points) Explain how this code can destroy memory beyond the end of
  the array.
\end{enumerate}

\begin{verbatim}



\end{verbatim}

\begin{enumerate}
\def\labelenumi{\alph{enumi}.}
\setcounter{enumi}{1}
\tightlist
\item
  (10 points) Modify this loop to protect against inadvertently changing
  memory beyond the end of the array.
\end{enumerate}

\begin{verbatim}



\end{verbatim}

\begin{enumerate}
\def\labelenumi{\arabic{enumi}.}
\setcounter{enumi}{3}
\tightlist
\item
  (20 points) Show the output produced when the following code (entire
  program not shown) executes. \emph{If there is an error in this code},
  point it out and explain why it is not correct. You are encouraged to
  also try to compile this to verify your results.
\end{enumerate}

\begin{Shaded}
\begin{Highlighting}[]
\NormalTok{string name = }\StringTok{"Porcupine Tree"}\NormalTok{; }
\NormalTok{cout \textless{}\textless{} }\StringTok{"NAME = "}\NormalTok{ + name \textless{}\textless{} endl; }
\NormalTok{cout \textless{}\textless{} name.length() \textless{}\textless{} endl; }
\NormalTok{name.erase(}\DecValTok{8}\NormalTok{, }\DecValTok{6}\NormalTok{);}
\NormalTok{cout \textless{}\textless{} name \textless{}\textless{} endl;}
\NormalTok{name.append(}\StringTok{"Dean WD Morgan"}\NormalTok{);}
\NormalTok{cout \textless{}\textless{} name \textless{}\textless{} endl;}
\NormalTok{name.insert(}\DecValTok{22}\NormalTok{, }\StringTok{"@TWD"}\NormalTok{);}
\NormalTok{cout \textless{}\textless{} name \textless{}\textless{} endl;}
\NormalTok{name.replace(}\DecValTok{23}\NormalTok{, }\DecValTok{3}\NormalTok{, }\StringTok{"The WD"}\NormalTok{);}
\NormalTok{cout \textless{}\textless{} name \textless{}\textless{} endl;}
\NormalTok{cout \textless{}\textless{} name.find(}\StringTok{"WD"}\NormalTok{) \textless{}\textless{} endl;}
\NormalTok{cout \textless{}\textless{} name.rfind(}\StringTok{"WD"}\NormalTok{) \textless{}\textless{} endl;}
\NormalTok{cout \textless{}\textless{} name.rfind(}\StringTok{"cupi"}\NormalTok{) \textless{}\textless{} endl;}
\ControlFlowTok{for}\NormalTok{ (}\DataTypeTok{int}\NormalTok{ i = name.length(); i \textgreater{} }\DecValTok{20}\NormalTok{; i{-}{-}) \{}
\NormalTok{   cout \textless{}\textless{} name[i{-}}\DecValTok{1}\NormalTok{];}
\NormalTok{   cout \textless{}\textless{} endl;}
\NormalTok{\}}
\end{Highlighting}
\end{Shaded}

\begin{verbatim}
\end{verbatim}

\pagebreak

\begin{enumerate}
\def\labelenumi{\arabic{enumi}.}
\setcounter{enumi}{4}
\tightlist
\item
  (20 points) Write the full definition of a function called
  \texttt{FunString()} that takes a string argument and does 2 things:
  (1) it prints the \emph{second} half of the string backwards (while
  still printing the first half normally), and (2) it reports on how
  many words the original string has (assume a word is separated with
  space characters). For example, if the argument is ``All the
  strings'', the function should print out ``All thesgnirts'' on one
  line and then the number \texttt{3} on the next line.
\end{enumerate}

\begin{verbatim}
\end{verbatim}

\pagebreak

\begin{enumerate}
\def\labelenumi{\arabic{enumi}.}
\setcounter{enumi}{5}
\tightlist
\item
  (20 points) Write a full definition for a function called
  \texttt{IsLoud()} that takes in a string argument and checks if each
  character in the string is an uppercase character \emph{or} a
  \texttt{\textquotesingle{}!\textquotesingle{}}. If all characters pass
  this test, then the function returns true, otherwise it returns false.
\end{enumerate}

\begin{verbatim}
\end{verbatim}

\end{document}
