% Options for packages loaded elsewhere
\PassOptionsToPackage{unicode}{hyperref}
\PassOptionsToPackage{hyphens}{url}
%
\documentclass[
]{article}
\usepackage{amsmath,amssymb}
\usepackage{lmodern}
\usepackage{ifxetex,ifluatex}
\ifnum 0\ifxetex 1\fi\ifluatex 1\fi=0 % if pdftex
  \usepackage[T1]{fontenc}
  \usepackage[utf8]{inputenc}
  \usepackage{textcomp} % provide euro and other symbols
\else % if luatex or xetex
  \usepackage{unicode-math}
  \defaultfontfeatures{Scale=MatchLowercase}
  \defaultfontfeatures[\rmfamily]{Ligatures=TeX,Scale=1}
\fi
% Use upquote if available, for straight quotes in verbatim environments
\IfFileExists{upquote.sty}{\usepackage{upquote}}{}
\IfFileExists{microtype.sty}{% use microtype if available
  \usepackage[]{microtype}
  \UseMicrotypeSet[protrusion]{basicmath} % disable protrusion for tt fonts
}{}
\makeatletter
\@ifundefined{KOMAClassName}{% if non-KOMA class
  \IfFileExists{parskip.sty}{%
    \usepackage{parskip}
  }{% else
    \setlength{\parindent}{0pt}
    \setlength{\parskip}{6pt plus 2pt minus 1pt}}
}{% if KOMA class
  \KOMAoptions{parskip=half}}
\makeatother
\usepackage{xcolor}
\IfFileExists{xurl.sty}{\usepackage{xurl}}{} % add URL line breaks if available
\IfFileExists{bookmark.sty}{\usepackage{bookmark}}{\usepackage{hyperref}}
\hypersetup{
  pdftitle={Homework 6: File I/O},
  pdfauthor={CS16 - Winter 2021},
  hidelinks,
  pdfcreator={LaTeX via pandoc}}
\urlstyle{same} % disable monospaced font for URLs
\usepackage{color}
\usepackage{fancyvrb}
\newcommand{\VerbBar}{|}
\newcommand{\VERB}{\Verb[commandchars=\\\{\}]}
\DefineVerbatimEnvironment{Highlighting}{Verbatim}{commandchars=\\\{\}}
% Add ',fontsize=\small' for more characters per line
\newenvironment{Shaded}{}{}
\newcommand{\AlertTok}[1]{\textcolor[rgb]{1.00,0.00,0.00}{\textbf{#1}}}
\newcommand{\AnnotationTok}[1]{\textcolor[rgb]{0.38,0.63,0.69}{\textbf{\textit{#1}}}}
\newcommand{\AttributeTok}[1]{\textcolor[rgb]{0.49,0.56,0.16}{#1}}
\newcommand{\BaseNTok}[1]{\textcolor[rgb]{0.25,0.63,0.44}{#1}}
\newcommand{\BuiltInTok}[1]{#1}
\newcommand{\CharTok}[1]{\textcolor[rgb]{0.25,0.44,0.63}{#1}}
\newcommand{\CommentTok}[1]{\textcolor[rgb]{0.38,0.63,0.69}{\textit{#1}}}
\newcommand{\CommentVarTok}[1]{\textcolor[rgb]{0.38,0.63,0.69}{\textbf{\textit{#1}}}}
\newcommand{\ConstantTok}[1]{\textcolor[rgb]{0.53,0.00,0.00}{#1}}
\newcommand{\ControlFlowTok}[1]{\textcolor[rgb]{0.00,0.44,0.13}{\textbf{#1}}}
\newcommand{\DataTypeTok}[1]{\textcolor[rgb]{0.56,0.13,0.00}{#1}}
\newcommand{\DecValTok}[1]{\textcolor[rgb]{0.25,0.63,0.44}{#1}}
\newcommand{\DocumentationTok}[1]{\textcolor[rgb]{0.73,0.13,0.13}{\textit{#1}}}
\newcommand{\ErrorTok}[1]{\textcolor[rgb]{1.00,0.00,0.00}{\textbf{#1}}}
\newcommand{\ExtensionTok}[1]{#1}
\newcommand{\FloatTok}[1]{\textcolor[rgb]{0.25,0.63,0.44}{#1}}
\newcommand{\FunctionTok}[1]{\textcolor[rgb]{0.02,0.16,0.49}{#1}}
\newcommand{\ImportTok}[1]{#1}
\newcommand{\InformationTok}[1]{\textcolor[rgb]{0.38,0.63,0.69}{\textbf{\textit{#1}}}}
\newcommand{\KeywordTok}[1]{\textcolor[rgb]{0.00,0.44,0.13}{\textbf{#1}}}
\newcommand{\NormalTok}[1]{#1}
\newcommand{\OperatorTok}[1]{\textcolor[rgb]{0.40,0.40,0.40}{#1}}
\newcommand{\OtherTok}[1]{\textcolor[rgb]{0.00,0.44,0.13}{#1}}
\newcommand{\PreprocessorTok}[1]{\textcolor[rgb]{0.74,0.48,0.00}{#1}}
\newcommand{\RegionMarkerTok}[1]{#1}
\newcommand{\SpecialCharTok}[1]{\textcolor[rgb]{0.25,0.44,0.63}{#1}}
\newcommand{\SpecialStringTok}[1]{\textcolor[rgb]{0.73,0.40,0.53}{#1}}
\newcommand{\StringTok}[1]{\textcolor[rgb]{0.25,0.44,0.63}{#1}}
\newcommand{\VariableTok}[1]{\textcolor[rgb]{0.10,0.09,0.49}{#1}}
\newcommand{\VerbatimStringTok}[1]{\textcolor[rgb]{0.25,0.44,0.63}{#1}}
\newcommand{\WarningTok}[1]{\textcolor[rgb]{0.38,0.63,0.69}{\textbf{\textit{#1}}}}
\usepackage{longtable,booktabs,array}
\usepackage{calc} % for calculating minipage widths
% Correct order of tables after \paragraph or \subparagraph
\usepackage{etoolbox}
\makeatletter
\patchcmd\longtable{\par}{\if@noskipsec\mbox{}\fi\par}{}{}
\makeatother
% Allow footnotes in longtable head/foot
\IfFileExists{footnotehyper.sty}{\usepackage{footnotehyper}}{\usepackage{footnote}}
\makesavenoteenv{longtable}
\setlength{\emergencystretch}{3em} % prevent overfull lines
\providecommand{\tightlist}{%
  \setlength{\itemsep}{0pt}\setlength{\parskip}{0pt}}
\setcounter{secnumdepth}{-\maxdimen} % remove section numbering
\ifluatex
  \usepackage{selnolig}  % disable illegal ligatures
\fi

\title{Homework 6: File I/O}
\author{CS16 - Winter 2021}
\date{}

\begin{document}
\maketitle

\begin{longtable}[]{@{}cl@{}}
\toprule
\endhead
\textbf{Due:} & Thursday, February 18, 2021 (11:59 PM
PST) \\ \addlinespace
\textbf{Points:} & 80 \\ \addlinespace
\textbf{Name:} &
\texttt{\_\_\_\_\_\_\_\_\_\_\_\_\_\_\_\_\_\_\_\_\_\_\_\_\_\_\_\_\_\_\_\_\_\_\_\_\_\_\_\_\_\_\_\_\_\_\_\_\_\_\_\_\_\_\_} \\ \addlinespace
\textbf{Homework buddy:} &
\texttt{\_\_\_\_\_\_\_\_\_\_\_\_\_\_\_\_\_\_\_\_\_\_\_\_\_\_\_\_\_\_\_\_\_\_\_\_\_\_\_\_\_\_\_\_\_\_\_\_\_\_\_\_\_\_\_} \\ \addlinespace
\bottomrule
\end{longtable}

\begin{itemize}
\tightlist
\item
  You may collaborate on this homework with \textbf{at most} one person,
  an optional ``homework buddy.''
\item
  \textbf{Submission instructions:} All questions are to be written
  (either by hand or typed) \emph{in the provided spaces} and turned in
  as a single PDF on Gradescope. If you submit handwritten solutions
  write legibly. We reserve the right to give 0 points to answers we
  cannot read. When you submit your answer on Gradescope, \textbf{be
  sure to select which portions of your answer correspond to which
  problem} and clearly mark on the page itself which problem you are
  answering. We reserve the right to give 0 points to submissions that
  fail to do this.
\end{itemize}

\pagenumbering{gobble}

\begin{enumerate}
\def\labelenumi{\arabic{enumi}.}
\tightlist
\item
  (10 points) When testing for end of file, the lecture (and the
  readings) mention two methods. What are they and when is each best
  used?
\end{enumerate}

\begin{verbatim}
\end{verbatim}

\pagebreak

\begin{enumerate}
\def\labelenumi{\arabic{enumi}.}
\setcounter{enumi}{1}
\tightlist
\item
  (10 points) Consider the following code snippet from a program where
  everything is set up correctly:
\end{enumerate}

\begin{Shaded}
\begin{Highlighting}[]
\NormalTok{cout \textless{}\textless{} “Enter a line of input: ”;}
\DataTypeTok{char}\NormalTok{ next;}
\ControlFlowTok{do}\NormalTok{ \{}
\NormalTok{   cin.get(next);}
\NormalTok{   cout \textless{}\textless{} next;}
\NormalTok{\} }\ControlFlowTok{while}\NormalTok{ (!(isdigit(next)) \&\& (next != }\CharTok{\textquotesingle{}}\SpecialCharTok{\textbackslash{}n}\CharTok{\textquotesingle{}}\NormalTok{));}
\NormalTok{cout \textless{}\textless{} “END!\textbackslash{}n”;}
\end{Highlighting}
\end{Shaded}

If the program dialog is as follows:

\begin{verbatim}
Enter a line of input: I'll see you 2nite at 9PM!
\end{verbatim}

then what will be the next line of output? \textbf{And explain why.}

\begin{verbatim}
\end{verbatim}

\pagebreak

\begin{enumerate}
\def\labelenumi{\arabic{enumi}.}
\setcounter{enumi}{2}
\tightlist
\item
  (10 points) I have a text file called ``t.txt'' that contains two
  entries: ``University of California'' on one line, and ``Computer
  Science Rules!'' on the next line.
\end{enumerate}

Show the output produced when the following code (entire program not
shown so assume all the necessary set ups are done correctly) is
executed \textbf{and explain why that is}. You are encouraged to also
try to compile this to verify your results.

\begin{Shaded}
\begin{Highlighting}[]
\NormalTok{ifstream tin;}
\DataTypeTok{char}\NormalTok{ c;}

\NormalTok{tin.open(}\StringTok{"t.txt"}\NormalTok{);}

\NormalTok{tin.get(c);}
\ControlFlowTok{while}\NormalTok{ (!tin.eof()) \{}
   \ControlFlowTok{if}\NormalTok{ ((c != }\CharTok{\textquotesingle{}e\textquotesingle{}}\NormalTok{) \&\& (c != }\CharTok{\textquotesingle{}C\textquotesingle{}}\NormalTok{)) \{}
\NormalTok{      cout \textless{}\textless{} c;}
\NormalTok{   \}}
\NormalTok{   tin.get(c);}
\NormalTok{\}}
\end{Highlighting}
\end{Shaded}

\begin{verbatim}
\end{verbatim}

\pagebreak

\begin{enumerate}
\def\labelenumi{\arabic{enumi}.}
\setcounter{enumi}{3}
\tightlist
\item
  (10 points) Complete the code below (there are several missing lines)
  such that the program reads a text file called ``MyInputs.txt'', which
  only contains double-type numbers separated by whitespaces. Of course,
  you don't know ahead of time how many numbers are in the file. You
  \textbf{must} use all the variables declared below, and you may add
  more variables as needed. Your program should print the average of
  these numbers, as indicated below.
\end{enumerate}

For example, if the text file contains this single line:

\begin{verbatim}
4.2 3.3 9.1 3.1 0 0 7.5 5.4 9.9 10
\end{verbatim}

Then the program should print out:

\begin{verbatim}
The average is: 5.25
\end{verbatim}

The code is as follows:

\begin{Shaded}
\begin{Highlighting}[]
\PreprocessorTok{\#include }\ImportTok{\textless{}iostream\textgreater{}}


\KeywordTok{using} \KeywordTok{namespace}\NormalTok{ std;}

\DataTypeTok{int}\NormalTok{ main ()}
\NormalTok{\{}
\NormalTok{   ifstream inf;}
   \DataTypeTok{double}\NormalTok{ num, sum(}\DecValTok{0}\NormalTok{), average;}
   \DataTypeTok{int}\NormalTok{ count = }\DecValTok{0}\NormalTok{;}











\NormalTok{   cout \textless{}\textless{} }\StringTok{"The average is: "}\NormalTok{ \textless{}\textless{} av \textless{}\textless{} endl;}



   \ControlFlowTok{return} \DecValTok{0}\NormalTok{;}
\NormalTok{\}}
\end{Highlighting}
\end{Shaded}

\pagebreak

\begin{enumerate}
\def\labelenumi{\arabic{enumi}.}
\setcounter{enumi}{4}
\tightlist
\item
  (40 points) Write a \textbf{function definition} for a function called
  \texttt{FindMedian()} that will read a file that just contains integer
  numbers that are separated by whitespaces and then finds the
  \emph{median} value. The median of a set of numbers is the number that
  has the same number of data elements greater than the number as there
  are less than the number.
\end{enumerate}

Requirements and Hints:

\begin{itemize}
\tightlist
\item
  You can assume that the input data text file will not have more than
  100 numbers in it (but it could have fewer).
\item
  You \textbf{have} to write the resulting median (just the integer and
  newline) in a separate output data file called
  \texttt{median\_output.dat}. Don't print anything to standard output.
\item
  The function does not have to check if the input or output files exist
  prior to reading/writing operations.
\item
  The function definition should only have two arguments: the input
  filename variable and the \texttt{ifstream} variable (see example
  function call below).
\item
  Hint 1: It is relatively easy to find the median in a set of sorted
  integers. And, yes, it's OK to define additional functions!
\item
  Hint 2: It matters if the number of integers is odd or even when
  finding the median. Think about how to address this.
\end{itemize}

The main function could look like this:

\begin{Shaded}
\begin{Highlighting}[]
\DataTypeTok{int}\NormalTok{ main() \{}
\NormalTok{   ifstream ifs;}
\NormalTok{   string fname = }\StringTok{"inputs.txt"}\NormalTok{; }\CommentTok{// could have other names too...}
\NormalTok{   FindMedian(fname, ifs);}
   \ControlFlowTok{return} \DecValTok{0}\NormalTok{;}
\NormalTok{\}}
\end{Highlighting}
\end{Shaded}

\textbf{Submission instructions:}

\begin{itemize}
\tightlist
\item
  You will submit this question (just this question) on Gradescope under
  ``HW 06 - Q5'' as a C++ file called \texttt{median.cpp} (cannot be
  named otherwise). This should only contain any and all function
  definitions you have, \textbf{except} for \texttt{main} (we'll supply
  that). We will test your submission with an autograder. This has the
  same due date as this homework.
\end{itemize}

\end{document}
